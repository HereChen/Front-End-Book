\section{前端系列之 JavaScript}\hypertarget{javascript}{}\label{javascript}

\subsection{JavaScript 组成}\hypertarget{javascript-}{}\label{javascript-}

一个完整的 JavaScript 实现是由以下 3 个不同部分组成的:

\begin{itemize}
\item 核心(ECMAScipt)
\item 文档对象模型(DOM)
\item 浏览器对象模型(BOM)
\end{itemize}

reference: \href{http://www.w3school.com.cn/js/pro\_js\_implement.asp}{JavaScript 实现},\href{http://segmentfault.com/a/1190000000654274}{JavaScript学习总结(三)BOM和DOM详解}

\subsection{类型、值和变量}\hypertarget{section}{}\label{section}

\subsubsection{数据类型}\hypertarget{section-1}{}\label{section-1}

\begin{itemize}
\item 基本数据类型:String 字符串;Number 数字;Boolean 布尔。
\item 复合数据类型:Object 对象;Array 数组。
\item 特殊数据类型:Null 空对象;Undefined 未定义。
\end{itemize}

reference: \href{https://msdn.microsoft.com/zh-cn/library/7wkd9z69(v=vs.94).aspx}{数据类型 (JavaScript), msdn}

\subsubsection{null, NaN, undefined}\hypertarget{null-nan-undefined}{}\label{null-nan-undefined}

JavaScript中有 6 个值为“假”: \texttt{false}, \texttt{null}, \texttt{undefined}, \texttt{0}, \texttt{''}(空字符串), \texttt{NaN}. 其中 \texttt{NaN} 是 JavaScript 中唯一不等于自身的值, 即 \texttt{NaN == NaN} 为 \texttt{false}.

\begin{verbatim}console.log( false == null )      // false
console.log( false == undefined ) // false
console.log( false == 0 )         // true
console.log( false == '' )        // true
console.log( false == NaN )       // false

console.log( null == undefined ) // true
console.log( null == 0 )         // false
console.log( null == '' )        // false
console.log( null == NaN )       // false

console.log( undefined == 0 )    // false
console.log( undefined == '' )   // false
console.log( undefined == NaN )  // false

console.log( 0 == '' )           // true
console.log( 0 == NaN )          // false
\end{verbatim}

对于 \texttt{===} 以上全为 \texttt{false}。对于 \texttt{==},以下几组为 \texttt{true}:\texttt{null} 和 \texttt{undefined};\texttt{false}、\texttt{0}、\texttt{''}。

reference: \href{http://www.cnblogs.com/snandy/p/3589517.html}{JavaScript中奇葩的假值},\href{http://www.ruanyifeng.com/blog/2014/03/undefined-vs-null.html}{阮一峰, undefined与null的区别, 2014}

\subsection{函数}\hypertarget{section-2}{}\label{section-2}

\subsubsection{常用函数}\hypertarget{section-3}{}\label{section-3}

\begin{itemize}
\item \texttt{string.slice(start,end)} 复制 string 中的一部分。
\item \texttt{string.indexOf(searchString, position)} 在 string 中查找 searchString。如果被找到,返回第一个匹配字符的位置,否则返回 -1。可选参数 position 可设置从 string 的某个指定的位置开始查找。
\item \texttt{object.hasOwnProperty(name)}
\item \texttt{array.splice(start, deleteCount, item)} 从 array 中移除一个或多个元素并用新的 item 替换他们。
\item \texttt{array.concat(item...)} 产生一个新数组,它包含一份 array 的前复制,并把一个或多个参数 item 附加在其后面。
\item \texttt{array.join(separator)} join 方法把一个 array 构成一个字符串。它先把 array 中的每个元素构造成一个字符串,接着用一个 separator 分隔符把它们连接在一起。
\item \texttt{array.pop()} 移除最后一个元素。
\item \texttt{array.push(item...)} 将一个或多个参数附加到数组的尾部。
\item \texttt{array.reverse()} 反转 array 元素的顺序。
\item \texttt{array.sort(camprefn)} 数组排序。
\item \texttt{array.shift()} 移除数组中的第一个元素并返回该元素。
\end{itemize}

reference: \href{http://book.douban.com/subject/11874748/}{Douglas Crockford, JavaScript语言精粹}

\subsubsection{Date}\hypertarget{date}{}\label{date}

\begin{verbatim}var t = new Date();
var tt = [
    t.getFullYear(), '年', // 不是 getYear()
    t.getMonth() + 1, '月',
    t.getDate(), '日', ' ',
    t.getHours(), '时',
    t.getMinutes(), '分',
    t.getSeconds(), '秒'
].join('');
console.log(tt); // 2015年10月30日 22时6分21秒
\end{verbatim}

注意 \texttt{getMonth()} 和 \texttt{getDay()} 都是从 0 开始的,需要加 1。

reference: \href{https://developer.mozilla.org/en-US/docs/Web/JavaScript/Reference/Global\_Objects/Date}{Date, mdn}

\subsubsection{typeof 和 instanceof}\hypertarget{typeof--instanceof}{}\label{typeof--instanceof}

( 1 ) typeof: \texttt{typeof operand}

typeof 操作符返回一个字符串,表示未经求值的操作数(unevaluated operand)的类型。typeof 只有一个实际应用场景,就是用来检测一个对象是否已经定义或者是否已经赋值。而这个应用却不是来检查对象的类型。除非为了检测一个变量是否已经定义,我们应尽量避免使用 typeof 操作符。

\begin{verbatim}typeof foo == 'undefined'  // 若 foo 未定义, 返回 true
\end{verbatim}

\begin{longtable}{|l|l|}   %  class="table"
\hline
类型 & 结构\\
\hline
Undefined & ``undefined''\\
Null & ``object''\\
布尔值 & ``boolean''\\
数值 & ``number''\\
字符串 & ``string''\\
Symbol (ECMAScript 6 新增) & ``symbol''\\
宿主对象(JS环境提供的,比如浏览器) & Implementation-dependent\\
函数对象 (implements [[Call]] in ECMA-262 terms) & ``function''\\
任何其他对象 & ``object''\\
\hline
\end{longtable}   %  class="table"

( 2 ) instanceof: \texttt{object instanceof constructor}

instanceof 运算符可以用来判断某个构造函数的 prototype 属性是否存在另外一个要检测对象的原型链上。(instanceof 运算符用来检测 constructor.prototype 是否存在于参数 object 的原型链上。)

\begin{verbatim}function C(){} // 定义一个构造函数
function D(){} // 定义另一个构造函数

var o = new C();
o instanceof C; // true,因为:Object.getPrototypeOf(o) === C.prototype
o instanceof D; // false,因为D.prototype不在o的原型链上
o instanceof Object; // true,因为Object.prototype.isPrototypeOf(o)返回true
C.prototype instanceof Object // true,同上

C.prototype = {};
var o2 = new C();
o2 instanceof C; // true
o instanceof C; // false,C.prototype指向了一个空对象,这个空对象不在o的原型链上.

D.prototype = new C();
var o3 = new D();
o3 instanceof D; // true
o3 instanceof C; // true
\end{verbatim}

reference: \href{http://segmentfault.com/a/1190000000730982}{typeof 和 instanceOf的区别},\href{https://developer.mozilla.org/zh-CN/docs/Web/JavaScript/Reference/Operators/instanceof}{instanceof},\href{https://developer.mozilla.org/zh-CN/docs/Web/JavaScript/Reference/Operators/typeof}{typeof}

\subsubsection{setInterval 和 setTimeout}\hypertarget{setinterval--settimeout}{}\label{setinterval--settimeout}

\begin{itemize}
\item \texttt{setTimeout} 只执行一次
\item \texttt{setInterval} 连续执行多次
\end{itemize}

\subsubsection{call 和 apply}\hypertarget{call--apply}{}\label{call--apply}

( 1 ) call: \texttt{fun.call(thisArg, arg1, arg2, ...)}

call() 方法在使用一个指定的this值和若干个指定的参数值的前提下调用某个函数或方法.

\begin{description}
\item[thisArg] 在fun函数运行时指定的this值。需要注意的是,指定的 this 值并不一定是该函数执行时真正的 this 值,如果这个函数处于非严格模式下,则指定为 null 和 undefined 的 this 值会自动指向全局对象(浏览器中就是 window 对象),同时值为原始值(数字,字符串,布尔值)的 this 会指向该原始值的自动包装对象。



\item[arg1, arg2, \ldots{}] 指定的参数列表。


/* 将函数的参数 arguments 转换为数组 */
  function listFirst()\{
      // this 指向 arguments
      // 下面的语句相当于 arguments.slice(0),但由于 arguments 不是数组,不能直接调用 slice 方法
      var arr = Array.prototype.slice.call(arguments, 0);
      for (var i=0; i \textless{} arr.length; i++)\{
          console.log(arr[i]);
      \}
  \}

listFirst(1,2,3); // 调用, 输出 1,2,3
\end{description}

( 2 ) apply: \texttt{fun.apply(thisArg, [argsArray])}

apply() 方法在指定 this 值和参数(参数以数组或类数组对象的形式存在)的情况下调用某个函数。

\begin{description}
\item[thisArg] 在 fun 函数运行时指定的 this 值。需要注意的是,指定的 this 值并不一定是该函数执行时真正的 this 值,如果这个函数处于非严格模式下,则指定为 null 或 undefined 时会自动指向全局对象(浏览器中就是window对象),同时值为原始值(数字,字符串,布尔值)的 this 会指向该原始值的自动包装对象。



\item[argsArray] 一个数组或者类数组对象,其中的数组元素将作为单独的参数传给 fun 函数。如果该参数的值为null 或 undefined,则表示不需要传入任何参数。从ECMAScript 5 开始可以使用类数组对象。浏览器兼容性请参阅本文底部内容。
\end{description}

在调用一个存在的函数时,你可以为其指定一个 this 对象,无需此参数时第一个参数可用 null(比如对于 add)。 this 指当前对象,也就是正在调用这个函数的对象。 使用 apply, 你可以只写一次这个方法然后在另一个对象中继承它,而不用在新对象中重复写该方法。

\begin{verbatim}/* 例一: 将函数的参数 arguments 转换为数组 */
function listFirst(){
    // this 指向 arguments
    var arr = Array.prototype.slice.apply(arguments, [0]);
    for (var i=0; i < arr.length; i++){
        console.log(arr[i]);
    }
}

listFirst(1,2,3); // 调用, 输出 1,2,3

/* 例二: push 一个数组 */
var arr1=new Array("1","2","3");
var arr2=new Array("4","5","6");
Array.prototype.push.apply(arr1,arr2); 
\end{verbatim}

reference: \href{https://developer.mozilla.org/zh-CN/docs/Web/JavaScript/Reference/Global\_Objects/Function/apply}{Function.prototype.apply()},\href{http://segmentfault.com/a/1190000000725712}{apply 和call的用法}

\subsubsection{new/构造函数}\hypertarget{new}{}\label{new}

构造函数只是一些使用 new 操作符时被调用的普通函数。使用 new 来调用函数,或者说发生构造函数调用时,会自动执行下面的操作。

\begin{enumerate}
\item 创建(或者说构造)一个全新的对象。
\item 这个新对象会被执行 [[prototype]] 链接。
\item 这个对象会绑定到函数调用的 this。
\item 如果函数没有返回其他对象,那么 new 表达式中的函数调用会自动返回这个新对象。
\end{enumerate}

\subsubsection{函数调用模式}\hypertarget{section-4}{}\label{section-4}

除了声明时定义的形式参数, 每个函数还接收两个附加的参数: this 和 arguments. 在 JavaScript 中一共有 4 种调用模式: 方法调用模式、函数调用模式、构造器调用模式和 apply 调用模式. 函数调用的模式不同, 对应的 this 值也会不同。

( 1 ) 方法调用模式

当一个函数被保存为对象的一个属性时,我们称它为一个方法。当一个方法被调用时,this 被绑定到该对象。

\begin{verbatim}// print 作为 obj 属性被保存,当 print 被调用时,this 指向 obj
var obj = {
    value: 'I am a string.',
    print: function (){
        console.log(this.value);
    }
}

obj.print(); // 调用
\end{verbatim}

( 2 ) 函数调用模式

当一个函数并非一个对象的属性时, 那么它就是被当做一个函数来调用。以此模式调用函数时,this 被绑定到全局对象。

\begin{verbatim}var func = function(){
    console.log('Hello here.');
}

func(); // 调用
\end{verbatim}

( 3 ) 构造器调用模式

如果在一个函数前面带上 new 来调用,那么背地里将会创建一个连接到该函数的 prototype 成员的新对象,同时 this 会绑定到那个新对象上。

\begin{verbatim}var Quo = function(string){
    this.status = string;
}
Quo.prototype.get_status = function(){
    return this.status;
}

var myQuo = new Quo('new Quo.') // 调用
myQuo.get_status(); // "new Quo."
\end{verbatim}

( 4 ) apply 调用模式

apply 方法让我们构建一个参数数组传递给调用函数。它允许我们选择 this 的值。apply 方法接收两个参数,第 1 个是要绑定给 this 的值,第 2 个就是一个参数数组.

reference: \href{http://book.douban.com/subject/3590768/}{《JavaScript语言精粹(修订版)》,第4章 函数}

\subsubsection{this}\hypertarget{this}{}\label{this}

this 在运行时绑定,它的上下文取决于函数调用时的各种条件。this 的绑定和函数声明的位置没有任何关系,只取决于函数的调用方式。

判断 this 的优先级,可以按照下面的顺序进行判断:

\begin{itemize}
\item 函数是否在 new 中调用 (new 绑定)?如果是的话 this 绑定的是新创建的对象。

var bar = new foo(); // 绑定 bar
\item 函数是否通过 call、apply (显示绑定) 或者硬绑定调用?如果是的话,this 绑定的是指定的对象。

var bar = foo.call(obj2); // 绑定 obj2
\item 函数是否在某个上下文对象中调用 (隐式调用)?如果是的话,this 绑定的是那个上下文对象。(调用时是否被某个对象拥有或包含,对象属性引用链中只有最顶层或者最后一层会影响调用位置 \texttt{obj1.obj2.foo(); // 绑定 obj2})

var bar = obj1.foo(); // 绑定 obj1
\item 如果都不是的话,使用默认绑定。如果在严格模式下,就绑定到 undefined,否则绑定到全局对象 (window)。

var bar = foo(); // 绑定 undefined 或 window
\end{itemize}

reference: \href{http://book.douban.com/subject/26351021/}{你不知道的JavaScript(上卷)}

\subsubsection{闭包}\hypertarget{section-5}{}\label{section-5}

当函数可以记住并访问所在的词法作用域时,就产生了闭包,即使函数是在当前词法作用域之外执行。(闭包是发生在定义时的。)

\begin{verbatim}// foo() 定义的中括号内就是 bar 的词法作用域
function foo() {
  var a = 2;
  function bar() {
    console.log( a );
  }
  return bar;
}
var baz = foo();
baz(); // 2, 这就是闭包, 用到变量 a

var fn;
function foo() {
  var a = 2;
  function baz() {
    console.log( a ); // 2
  }
  fn = baz; // 将 baz 赋值给全局变量
  // 调用 fn 相当于执行的 baz, 而其词法作用域在 foo() 定义函数内.
}

function bar(fn) {
  fn(); // 这就是闭包, 用到变量 a
}

foo();
bar(); // 2
\end{verbatim}

reference: \href{http://book.douban.com/subject/26351021/}{你不知道的JavaScript(上卷)}

\subsection{继承方法}\hypertarget{section-6}{}\label{section-6}

\begin{itemize}
\item 原型链继承
\item 构造继承
\item 实例继承
\item 拷贝继承
\end{itemize}

reference: \href{https://developer.mozilla.org/zh-CN/docs/Web/JavaScript/Inheritance\_and\_the\_prototype\_chain}{继承与原型链, mdn}, \href{http://raychase.iteye.com/blog/1337415}{RayChase, JavaScript实现继承的几种方式}

\subsection{正则表达式}\hypertarget{section-7}{}\label{section-7}

( 1 ) 特殊字符

\begin{itemize}
\item \texttt{\textbackslash{}d} 任意一个数字,等价于 \texttt{[0-9]}
\item \texttt{\textbackslash{}D} 任意一个非数字,等价于 \texttt{[\^{}0-9]}
\item \texttt{\textbackslash{}w} 任意一个字母、数字或下划线字符,等价于 \texttt{[a-zA-Z\_]}
\item \texttt{\textbackslash{}W} 任意一个非字母、数字和下划线字符,等价于 \texttt{[\^{}a-zA-Z\_]}
\item \texttt{\textbackslash{}s} 任意一个空白字符,包括换页符、换行符、回车符、制表符和垂直制表符,等价于 \texttt{[\textbackslash{}f\textbackslash{}n\textbackslash{}r\textbackslash{}t\textbackslash{}v]}
\item \texttt{\textbackslash{}S} 任意一个非空白符,等价于 \texttt{[\^{}\textbackslash{}f\textbackslash{}n\textbackslash{}r\textbackslash{}t\textbackslash{}v]}
\item \texttt{.} 换行和回车以外的任意一个字符,等价于 \texttt{[\^{}\textbackslash{}n\textbackslash{}r]}
\end{itemize}

( 2 ) 次数匹配

\begin{itemize}
\item \texttt{?} 最多一次 (零次或一次)
\item \texttt{+} 至少一次
\item \texttt{*} 任意次
\item \texttt{\{n\}} 只能出现 n 次
\item \texttt{\{n,m\}} 至少 n 次,最多 m 次
\end{itemize}

reference: \href{http://www.nowamagic.net/librarys/veda/detail/1283}{正则总结:JavaScript中的正则表达式}

\subsection{ES5 新特性}\hypertarget{es5-}{}\label{es5-}

\begin{quote}
新功能包括:原生 JSON 对象、继承的方法、高级属性的定义以及引入严格模式。
\end{quote}

reference: \href{http://segmentfault.com/a/1190000003493604}{梦禅, ECMAScript各版本简介及特性, segmentfault}

\subsection{ES6 新特性}\hypertarget{es6-}{}\label{es6-}

模块,类,块级作用域,Promise,生成器\ldots{}

\subsection{Ajax}\hypertarget{ajax}{}\label{ajax}

\subsubsection{Ajax 实现}\hypertarget{ajax-}{}\label{ajax-}

( 1 ) 原生 Ajax 实现: GET

\begin{verbatim}var xmlhttp = new XMLHttpRequest();
xmlhttp.open("GET","test1.txt",true);
xmlhttp.send();
\end{verbatim}

( 2 ) 原生 Ajax 实现: POST

\begin{verbatim}var xmlhttp = new XMLHttpRequest();
xmlhttp.onreadystatechange = function(){
  if (xmlhttp.readyState==4 && xmlhttp.status==200){
    document.getElementById("myDiv").innerHTML=xmlhttp.responseText;
  }
}
xmlhttp.open("POST","ajax_test.asp",true);
xmlhttp.setRequestHeader("Content-type","application/x-www-form-urlencoded");
xmlhttp.send("fname=Bill&lname=Gates");
\end{verbatim}

reference: \href{http://segmentfault.com/a/1190000003096293}{原生JS与jQuery对AJAX的实现},\href{http://www.w3school.com.cn/ajax/ajax\_xmlhttprequest\_send.asp}{AJAX - 向服务器发送请求}

\subsubsection{Ajax 的 5 个状态 (readyState)}\hypertarget{ajax--5--readystate}{}\label{ajax--5--readystate}

新建对象 -\textgreater{} 建立连接 -\textgreater{} 接收响应原始数据 -\textgreater{} 解析原始数据 -\textgreater{} 响应就绪(待后续处理)

( 1 ) 0: 请求未初始化

此阶段确认 XMLHttpRequest 对象是否创建,并为调用 open() 方法进行未初始化作好准备。值为 0 表示对象已经存在,否则浏览器会报错--对象不存在。

( 2 ) 1: 服务器连接已建立

此阶段对 XMLHttpRequest 对象进行初始化,即调用 open() 方法,根据参数 (method,url,true)完成对象状态的设置。并调用 send() 方法开始向服务端发送请求。值为 1 表示正在向服务端发送请求。

( 3 ) 2: 请求已接收

此阶段接收服务器端的响应数据。但获得的还只是服务端响应的原始数据,并不能直接在客户端使用。值为2表示已经接收完全部响应数据。并为下一阶段对数据解析作好准备。

( 4 ) 3: 请求处理中

此阶段解析接收到的服务器端响应数据。即根据服务器端响应头部返回的MIME类型把数据转换成能通过 responseBody、responseText 或 responseXML 属性存取的格式,为在客户端调用作好准备。状态 3 表示正在解析数据。

( 5 ) 4: 请求已完成,且响应已就绪

此阶段确认全部数据都已经解析为客户端可用的格式,解析已经完成。值为 4 表示数据解析完毕,可以通过 XMLHttpRequest 对象的相应属性取得数据。

reference: \href{http://blog.163.com/freestyle\_le/blog/static/183279448201269112527311/}{Panda, Ajax readyState的五种状态, LOFTER}

\subsection{文档对象 DOM}\hypertarget{dom}{}\label{dom}

\subsubsection{浏览器事件冒泡和捕获}\hypertarget{section-8}{}\label{section-8}

事件分为三个阶段:

\begin{itemize}
\item 捕获阶段
\item 目标阶段
\item 冒泡阶段
\end{itemize}

TODO:IE 和 w3c 标准的区别;阻止事件传播(捕获,冒泡) \texttt{e.stopPropagation()};阻止事件默认行为 \texttt{e.preventDefault()}。

\begin{verbatim}function registerEventHandler(node, event, handler) {
  if (typeof node.addEventListener == "function")
    node.addEventListener(event, handler, false);
  else
    node.attachEvent("on" + event, handler);
}

registerEventHandler(button, "click", function(){print("Click (2)");});
\end{verbatim}

reference: \href{http://segmentfault.com/a/1190000003482372}{本期节目, 浏览器事件模型中捕获阶段、目标阶段、冒泡阶段实例详解, segmentfault, 2015.8}

\subsection{浏览器对象 BOM}\hypertarget{bom}{}\label{bom}

\subsubsection{弹框}\hypertarget{section-9}{}\label{section-9}

\begin{itemize}
\item \texttt{prompt(text,defaultText)} 提示用户输入的对话框。\texttt{text} 对话框中显示的纯文本,\texttt{defaultText} 默认的输入文本。返回值为输入文本。
\item \texttt{alert(message)} 警告框。\texttt{message} 对话框中要实现的纯文本。
\item \texttt{confirm(message)} 显示一个带有指定消息和 OK 及取消按钮的对话框。\texttt{message} 对话框中显示的纯文本。点击确认返回 true,点击取消返回 false。
\end{itemize}

reference: \href{http://www.w3school.com.cn/jsref/met\_win\_prompt.asp}{http://www.w3school.com.cn/jsref/met\_win\_prompt.asp}

\subsubsection{localStorage}\hypertarget{localstorage}{}\label{localstorage}

\begin{quote}
存储在浏览器中的数据,如 localStorage 和 IndexedDB,以源进行分割。每个源都拥有自己单独的存储空间,一个源中的Javascript脚本不能对属于其它源的数据进行读写操作。
\end{quote}

localStorage 在当前源下设置的值,只能在当前源下查看。

\begin{verbatim}localStorage.setItem('key',value)   // 设置值, 或 localStorage.key = value
localStorage.getItem('key')         // 获取值, 或 localStorage.key
localStorage.removeItem('key')      // 删除值
localStorage.clear()                // 清空 localStorage
\end{verbatim}

reference: \href{https://developer.mozilla.org/zh-CN/docs/Web/Security/Same-origin\_policy}{JavaScript 的同源策略, MDN}, \href{https://developer.mozilla.org/en-US/docs/Web/API/Storage/LocalStorage}{localStorage, MDN}, \href{http://wikieswan.github.io/javascript/2015/04/03/html5-api-localstorage/}{杜若, localStorage 介绍}

\subsection{jQuery}\hypertarget{jquery}{}\label{jquery}

\subsubsection{jQuery 知识结构}\hypertarget{jquery-}{}\label{jquery-}

\begin{itemize}
\item jQuery 基础: 选择器
\item jQuery 效果: hide, show, toggle, fadeIn, fadeOut, animate
\item jQuery 操作 HTML: text, html, val, attr, append, after, prepend, before, remove, empty
\item jQuery 遍历 Dom: parent, parents, children, find
\end{itemize}

\subsubsection{jQuery 和 Dom 对象相互转换}\hypertarget{jquery--dom-}{}\label{jquery--dom-}

( 1 ) jQuery 对象转换成 Dom 对象: 下标和 get 方法

\begin{verbatim}/* 方法一: 下标 */
var $v = $("#v") ; //jQuery对象
var v  = $v[0];    //DOM对象

/* 方法二: get */
var $v = $("#v");   //jQuery对象
var v  = $v.get(0); //DOM对象
\end{verbatim}

( 2 ) Dom 对象转 jQuery 对象: 通过 \texttt{\$()} 把 Dom 对象包装起来实现

\begin{verbatim}var v  = document.getElementById("v"); //DOM对象
var $v = $(v); //jQuery对象
\end{verbatim}

reference: \href{http://segmentfault.com/a/1190000003710344}{jQuery对象与dom对象的区别与相互转换}

\subsubsection{jQuery 命名冲突解决}\hypertarget{jquery--1}{}\label{jquery--1}

( 1 ) 使用 \texttt{jQuery.noConflict()}

\begin{verbatim}jQuery.noConflict(); 
// 之后使用 jQuery 调用, jQuery("#id").methodname()
\end{verbatim}

( 2 ) 自定义别名

\begin{verbatim}var $j = jQuery.noConflict();
// $j("#id").methodname();
\end{verbatim}

( 3 ) 传入参数,继续使用 \texttt{\$}

\begin{verbatim}jQuery.noConflict(); 
(function($){ 
    $("#id").methodname();
})(jQuery)
\end{verbatim}

reference: \href{http://ued.taobao.org/blog/2013/03/jquery-noconflict/}{邦彦, 谈谈 jQuery 中的防冲突(noConflict)机制, TaoBaoUED}, \href{http://www.cnblogs.com/RascallySnake/archive/2010/05/07/1729417.html}{RascallySnake, Jquery的\$命名冲突, cnblogs}, \href{http://www.cnblogs.com/ForEvErNoME/archive/2012/03/15/2398659.html}{TerryChen, Jquery命名冲突解决的五种方案, cnblogs}

\subsection{解决方案}\hypertarget{section-10}{}\label{section-10}

跨域,客户端存储

\subsubsection{跨域}\hypertarget{section-11}{}\label{section-11}

\begin{itemize}
\item \href{http://www.alloyteam.com/2013/11/the-second-version-universal-solution-iframe-cross-domain-communication/}{ TAT.Johnny, iframe跨域通信的通用解决方案, alloyteam}
\item \href{http://segmentfault.com/a/1190000003642057}{JasonKidd,「JavaScript」四种跨域方式详解, segmentfault}
\end{itemize}

\subsection{其他}\hypertarget{section-12}{}\label{section-12}

\begin{enumerate}
\item switch 以 \texttt{===} 匹配。
\item 函数会首先被提升,然后才是变量。
\end{enumerate}
