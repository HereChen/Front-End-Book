\section{前端系列之 CSS}\hypertarget{css}{}\label{css}

定位, 盒模型, 浮动, 媒体查询(自适应布局).

\subsection{定位}\hypertarget{section}{}\label{section}

CSS 有三种基本的定位机制:普通流、浮动和绝对定位。任何元素都可以定位,不过绝对或固定元素会生成一个块级框,而不论该元素本身是什么类型。

\textbf{说明}

\begin{longtable}{|l|l|}   %  class="table"
\hline
默认值: & static\\
继承性: & no\\
版本: & CSS2\\
JavaScript 语法: & object.style.position=''absolute''\\
\hline
\end{longtable}   %  class="table"

\begin{longtable}{|l|l|}   %  class="table"
\hline
值 & 描述\\
\hline
absolute & 生成绝对定位的元素,相对于 static 定位以外的第一个父元素进行定位。元素的位置通过 ``left'', ``top'', ``right'' 以及 ``bottom'' 属性进行规定。\\
fixed & 生成绝对定位的元素,相对于浏览器窗口进行定位。元素的位置通过 ``left'', ``top'', ``right'' 以及 ``bottom'' 属性进行规定。\\
relative & 生成相对定位的元素,相对于其正常位置进行定位。因此,''left:20'' 会向元素的 LEFT 位置添加 20 像素。\\
static & 默认值。没有定位,元素出现在正常的流中(忽略 top, bottom, left, right 或者 z-index 声明)。\\
inherit & 规定应该从父元素继承 position 属性的值。\\
\hline
\end{longtable}   %  class="table"

\begin{itemize}
\item 如果一个标签的位置是绝对的,它又不在其他任何设定了 absolute、relative 或 fixed 定位的标签里面,那它就是相对于浏览器窗口进行定位。
\item 如果一个标签处在另一个设定了 absolute、relative 或 fixed 定位的标签里面,那它就是相对于另一个元素的边沿进行定位。
\end{itemize}

reference: \href{http://www.w3school.com.cn/cssref/pr\_class\_position.asp}{http://www.w3school.com.cn/cssref/pr\_class\_position.asp},\href{http://book.douban.com/subject/4861462/}{CSS实战手册(第2版)}

\subsection{盒模型}\hypertarget{section-1}{}\label{section-1}

IE 盒模型和标准存在区别。

\subsection{优先级}\hypertarget{section-2}{}\label{section-2}

优先级\textbf{从高到低},分三个层次描述。

( 1 ) 作者/用户/浏览器样式。作者样式指网页本身的样式,或者开发者编写的样式;用户样式指浏览网页的用户自己添加的样式表(通过浏览器设置);浏览器样式指浏览器提供的默认样式。

\begin{itemize}
\item 标有 \texttt{!important} 的用户样式。
\item 标有 \texttt{!important} 的作者样式。
\item 作者样式。
\item 用户样式。
\item 浏览器样式。
\end{itemize}

( 2 ) 作者样式:内部样式(internal)、内联样式(inline)、外部样式(external)。内部样式 style 标签中声明的样式;内联样式指元素属性 style 中的样式;外部样式指通过 link 链接的外部文件中样式。

\begin{itemize}
\item 内联样式(行内样式)
\item 内部样式
\item 外部样式
\end{itemize}

外部样式和内部样式在优先级相同的情况下,后定义的会覆盖先定义的。

( 3 ) CSS 选择器

\begin{itemize}
\item ID 选择器:\texttt{\#idname}
\item 伪类:\texttt{:hover}
\item 属性选择器:\texttt{input[type="text"]}
\item 类选择器:\texttt{.classname}
\item 元素(类型)选择器(包括伪元素):\texttt{input}, \texttt{:after}
\item 通用选择器:\texttt{*}
\end{itemize}

reference: \href{http://book.douban.com/subject/4736167/}{Cameron Moll, 精通CSS(第2版)},\href{https://developer.mozilla.org/zh-CN/docs/Web/CSS/Specificity}{优先级, mdn}, \href{https://vineetgupta22.wordpress.com/2011/07/09/inline-vs-internal-vs-external-css/}{INLINE VS INTERNAL VS EXTERNAL CSS}

\subsection{link 和 @import}\hypertarget{link--import}{}\label{link--import}

\begin{itemize}
\item \texttt{@import} 和 link 混用时,可能会出现不同时下载的情况。
\item 考虑两者混合使用的浏览器实现的不一样(下载次序)。
\end{itemize}

reference: \href{http://www.stevesouders.com/blog/2009/04/09/dont-use-import/}{Steve Souders, don’t use @import}, \href{http://www.dreamdu.com/blog/2007/05/11/css\_link\_import/}{外部引用CSS中 link与@import的区别}

\subsection{浮动和文档流}\hypertarget{section-3}{}\label{section-3}

\subsection{两栏等高布局}\hypertarget{section-4}{}\label{section-4}

父元素设置 \texttt{overflow:hidden;},子元素设置 \texttt{padding-bottom:10000px; margin-bottom:-10000px;}。
\texttt{overflow:hidden;} 截除超出的高度,\texttt{margin-bottom:-10000px;} 抵消较高内容超出的 \texttt{padding}。

\begin{verbatim}<div id="fa">
  <div class="col">
    <p>1231321321</p>
    <p>1313213</p>
    <p>1313213</p>
    <p>1313213</p>
    <p>1313213</p>
    <p>1313213</p>
  </div>
  <div class="col">45645456456</div>
</div>

#fa {
  width: 800px;
  margin: 0 auto;
  background-color: #1524e5;
  overflow: hidden;
}

.col {
  float: left;
  width: 50%;
  padding-bottom: 10000px;
  margin-bottom: -10000px;
}

.col:first-child {
  background-color: #34ef34;
}
.col:last-child {
  background-color: #ef34ef;
}
\end{verbatim}

refernce: \href{http://segmentfault.com/a/1190000000625584}{CSS/两栏并列等高布局}

\subsection{动画}\hypertarget{section-5}{}\label{section-5}

动画的实现方式可以分为几种:CSS3 原生实现;JavaScript 实现 (操作 CSS、canavs、svg);动画文件 (gif, flash)。

CSS3 的实现可以通过 @keyframes 和 animation 完成,@keyframes 定义动画,animation 调用动画。

\begin{verbatim}/* 定义 */
@keyframes changebackcolor{
  from {background: red;}
  to {background: yellow;}
}
/* 调用 */
div{
  animation: changebackcolor 5s;
}
\end{verbatim}

reference: \href{http://www.w3school.com.cn/css3/css3\_animation.asp}{http://www.w3school.com.cn/css3/css3\_animation.asp}

\subsection{垂直居中和水平居中}\hypertarget{section-6}{}\label{section-6}

( 1 ) 块元素垂直居中:transform: translateY(-50\%)。此方案存在兼容性问题(Firefox 43 不支持, Chrome 46 支持)

\begin{verbatim}width: 250px;
height: 250px;
position: relative;
top: 50%;
transform: translateY(-50%);
\end{verbatim}

( 2 ) 块元素垂直居中:绝对定位(top: 50\%)

\begin{verbatim}width: 250px;
height: 250px;
position: absolute;
top: 50%;
margin-top: -125px;
\end{verbatim}

( 3 ) 内联元素垂直居中

\begin{verbatim}/* 方案一: 块元素内容会居中, 需要设置高度 */
display: table-cell;
vertical-align: middle;

/* 方案二: 一般单行文本的上下居中 */
line-height: 50px;
\end{verbatim}

( 4 ) 块元素的水平居中:margin: 0 auto

\begin{verbatim}width: 100px;
margin-left: auto;
margin-right: auto
\end{verbatim}
