\section{前端系列之浏览器兼容性}\hypertarget{section}{}\label{section}

\subsection{IE 条件注释}\hypertarget{ie-}{}\label{ie-}

可以添加条件语句检测浏览器,针对性的编写样式。

\begin{verbatim}<!--[if IE 8]>
<body class="ie8">
<![endif]-->
<!--[if !IE]>
<body class="notie">
<![endif]-->
\end{verbatim}

reference: \href{https://msdn.microsoft.com/en-us/library/ms537512(v=vs.85).aspx}{About conditional comments, msdn}

\subsection{IE 支持 CSS}\hypertarget{ie--css}{}\label{ie--css}

使用条件语句判断 IE,并用 CSS 表达式解决 (一般情况下不推荐采用表达式,效率低)。在 IE8 标准模式中,不支持 CSS Expression。

\begin{verbatim}/* 正常 IE6/IE7/IE8 不支持*/
min-height:50px;
/* 表达式 */
_height:expression((parseInt(this.currentStyle.height) < 50) ? 50 : this.clientHeight)
\end{verbatim}

\subsection{IE6/IE7/IE8 支持 html5 新标签}\hypertarget{ie6ie7ie8--html5-}{}\label{ie6ie7ie8--html5-}

( 1 ) 创建标签

\begin{verbatim}document.createElement('section'); // 其他标签一样处理
\end{verbatim}

( 2 ) 使用方案

\begin{itemize}
\item html5shiv:\href{https://github.com/aFarkas/html5shiv/}{https://github.com/aFarkas/html5shiv/} (用的也是 createElement())
\item modernizr:\href{https://github.com/modernizr/modernizr/}{https://github.com/modernizr/modernizr/} (这个功能要多一点,还兼顾 CSS3,\href{http://www.osmn00.com/translation/221.html}{更多})
\end{itemize}

\subsection{IE 兼容性测试}\hypertarget{ie--1}{}\label{ie--1}

采用 \href{https://dev.modern.ie/}{Modern.IE} 提供的\href{https://dev.modern.ie/tools/vms/windows/}{虚拟机}测试。下载 \href{https://www.virtualbox.org/}{virtualbox} (免费) 或者 \href{http://www.vmware.com/}{VMware} (收费),然后导入对应的下载包(不同系统 IE 版本不同)。

\subsection{不支持 JavaScript}\hypertarget{javascript}{}\label{javascript}

noscript 元素用来定义在脚本未被执行时的替代内容(文本)。此标签可被用于可识别 \texttt{\textless{}script\textgreater{}} 标签但无法支持其中的脚本的浏览器。

\begin{verbatim}<noscript>Your browser does not support JavaScript!</noscript>
\end{verbatim}

reference: \href{http://www.w3school.com.cn/tags/tag\_noscript.asp}{HTML noscript 标签, w3school}

\subsection{浏览器 hack}\hypertarget{hack}{}\label{hack}

以 IE 为例,展示几个 CSS hack 方法,更多的见参考链接。

( 1 ) IE6

\begin{verbatim}.selector { _property: value; }
.selector { -property: value; }
\end{verbatim}

( 2 ) IE \textless{}= 7 ( \texttt{! \$ \& * ( ) = \% + @ , . / ` [ ] \# \ensuremath{\sim} ? : \textless{} \textgreater{} \textbar{}} )

\begin{verbatim}.selector { !property: value; }
.selector { $property: value; }
.selector { &property: value; }
.selector { *property: value; }
.selector { )property: value; }
.selector { =property: value; }
.selector { %property: value; }
.selector { +property: value; }
.selector { @property: value; }
.selector { ,property: value; }
.selector { .property: value; }
.selector { /property: value; }
.selector { `property: value; }
.selector { ]property: value; }
.selector { #property: value; }
.selector { ~property: value; }
.selector { ?property: value; }
.selector { :property: value; }
.selector { |property: value; }
\end{verbatim}

( 3 ) IE 6-8

\begin{verbatim}.selector { property: value\9; }
.selector { property/*\**/: value\9; }
\end{verbatim}

reference: \href{http://browserhacks.com/}{http://browserhacks.com/}

\subsection{CSS3 前缀}\hypertarget{css3-}{}\label{css3-}

虽然目前 CSS3 得到广泛支持,但各个浏览器厂商对标准实现并不完全一样,可以对 CSS3 使用前缀。建议把特殊的 CSS 语句放在前面,一般的语句放置在后面。可以使用 \href{https://github.com/search?utf8=\%E2\%9C\%93\&q=Autoprefixer}{Autoprefixer} 自动添加前缀。

\begin{itemize}
\item \texttt{-moz-} Firefox,
\item \texttt{-webkit-} Safari, Chrome
\item \texttt{-o-} Opera (不过, Opera 和 Chrome 现在都采用 Blink 内核)
\item \texttt{-ms-} Internet Explorer
\end{itemize}

\subsection{浏览器检测}\hypertarget{section-1}{}\label{section-1}

通过 js 检测浏览器可以分为两种:依靠浏览器各自的特性检测;通过浏览器对象检测 \texttt{navigator.userAgent}。

( 1 ) navigator.userAgent

可直接查看此插件的 js 代码:\href{https://github.com/gabceb/jquery-browser-plugin}{jquery-browser-plugin}。实测了 IE、Firefox、Opera、Chrome,输出结果都是对的。

\begin{verbatim}<html>
<head>
    <meta charset="UTF-8">
    <title>Document</title>
    <script src="jquery.browser.js"></script>
</head>
<body>
<script>
    console.log("jQBrowser.webkit: " + jQBrowser.webkit);
    console.log("jQBrowser.mozilla: " + jQBrowser.mozilla);
    console.log("jQBrowser.msie: " + jQBrowser.msie);
    console.log("jQBrowser.version: " + jQBrowser.version);
</script>
</body>
</html>
\end{verbatim}

( 2 ) 通过特性检测判断:占位待续

reference: \href{http://www.ludou.org/2-way-to-detect-browser.html}{露兜博客, 检测访客浏览器的2种方法, 2009}

