\section{代码规范}\label{ux4ee3ux7801ux89c4ux8303}

\subsection{编辑器文本规范}\label{ux7f16ux8f91ux5668ux6587ux672cux89c4ux8303}

规范文件采用的换行符、缩进方式以及编码等等.

\begin{enumerate}
\def\labelenumi{\arabic{enumi}.}
\tightlist
\item
  EditorConfig: \url{http://editorconfig.org/}
\item
  VSCode 插件:
  \href{https://marketplace.visualstudio.com/items?itemName=EditorConfig.EditorConfig}{EditorConfig
  for VS Code}, 可生成配置样本.
\end{enumerate}

\lstinline!.editorconfig!配置样例

\begin{lstlisting}
root = true

[*]
indent_style = space
indent_size = 2
charset = utf-8
end_of_line = lf
insert_final_newline = true
trim_trailing_whitespace = true
max_line_length = 80
\end{lstlisting}

\subsection{命名规范}\label{ux547dux540dux89c4ux8303}

\subsubsection{CSS 命名}\label{css-ux547dux540d}

BEM, SMACSS, OOCSS

\subsection{代码检查}\label{ux4ee3ux7801ux68c0ux67e5}

\subsubsection{CSS 格式化}\label{css-ux683cux5f0fux5316}

\begin{enumerate}
\def\labelenumi{\arabic{enumi}.}
\tightlist
\item
  CSScomb: \url{http://csscomb.com}
\item
  配置文件\lstinline!.csscomb.json!
  示例:\url{https://github.com/htmlacademy/codeguide/blob/master/csscomb.json}
\item
  VSCode 插件:
  \href{https://marketplace.visualstudio.com/items?itemName=mrmlnc.vscode-csscomb}{CSScomb}
\end{enumerate}

\subsubsection{JS
静态代码检查工具}\label{js-ux9759ux6001ux4ee3ux7801ux68c0ux67e5ux5de5ux5177}

\begin{enumerate}
\def\labelenumi{\arabic{enumi}.}
\tightlist
\item
  ESLint: \url{https://github.com/eslint/eslint}
\item
  JSLint: \url{https://github.com/jshint/jshint/}
\end{enumerate}

\subsubsection{JS 语法规范}\label{js-ux8bedux6cd5ux89c4ux8303}

\begin{enumerate}
\def\labelenumi{\arabic{enumi}.}
\tightlist
\item
  airbnb: \url{https://github.com/airbnb/javascript}
\item
  standard: \url{https://github.com/standard/standard}
\end{enumerate}

\section{兼容性问题解决}\label{ux517cux5bb9ux6027ux95eeux9898ux89e3ux51b3}

\subsection{IE: 盒模型}\label{ie-ux76d2ux6a21ux578b}

IE 默认情况下长宽包含 \lstinline!padding! 和 \lstinline!border!,
和其他浏览器的长宽存在区别, 建议添加 \lstinline!border-sizing! 属性.
保持多个浏览器的一致性.

\begin{lstlisting}
box-sizing: border-box;
\end{lstlisting}

\subsection{IE 8: map}\label{ie-8-map}

IE8 不支持 JavaScript 原生 map 函数,
可在任意地方加入如下的代码片段\footnote{\href{http://stackoverflow.com/questions/7350912/is-the-javascript-map-function-supported-in-ie8}{Is
  the javascript .map() function supported in IE8?}}.

\begin{lstlisting}
(function (fn) {
    if (!fn.map) fn.map = function (f) {
        var r = [];
        for (var i = 0; i < this.length; i++)
            if (this[i] !== undefined) r[i] = f(this[i]);
        return r
    }
    if (!fn.filter) fn.filter = function (f) {
        var r = [];
        for (var i = 0; i < this.length; i++)
            if (this[i] !== undefined && f(this[i])) r[i] = this[i];
        return r
    }
})(Array.prototype);
\end{lstlisting}

或者用 jQuery 的 map 函数.

\begin{lstlisting}
// array.map(function( ) { });
jQuery.map(array, function( ) {
}
\end{lstlisting}

\subsection{IE 8: fontawesome
图标显示为方块}\label{ie-8-fontawesome-ux56feux6807ux663eux793aux4e3aux65b9ux5757}

对于修饰性不影响功能的图标, 可以做降级处理, 仅在非 IE 或者 IE9+
(条件注释\footnote{\href{https://msdn.microsoft.com/en-us/library/ms537512(v=vs.85).aspx}{About
  conditional comments}})情况下引入 fontawesome 图标库.
(谷歌搜索了一堆方案都没用, 最后应用这种方式来解决).

\begin{lstlisting}[language=HTML]
<!--[if (gt IE 8) | !IE]><!-->
<link rel="stylesheet" href="font-awesome.min.css">
<!--<![endif]-->
\end{lstlisting}

\subsection{IE 10+
浏览器定位}\label{ie-10-ux6d4fux89c8ux5668ux5b9aux4f4d}

IE 10+ 不支持条件注释, 因此需要其他方式定位这些浏览器. 如果只增加 CSS,
可采用以下方式定位\footnote{\href{http://stackoverflow.com/questions/9900311/how-do-i-target-only-internet-explorer-10-for-certain-situations-like-internet-e/14916454\#14916454}{How
  do I target only Internet Explorer 10 for certain situations like
  Internet Explorer-specific CSS or Internet Explorer-specific
  JavaScript code?}}.

\begin{lstlisting}
/*IE 9+, 以及 Chrome*/
@media screen and (min-width:0\0) {
}

/*IE 10*/
@media all and (-ms-high-contrast: none), (-ms-high-contrast: active) {
}

/*Edge*/
@supports (-ms-accelerator:true) {
}
\end{lstlisting}

另一种方式是 JavaScript 检测浏览器版本, 在 \lstinline!body!
标签为特定浏览器添加 \lstinline!class! 属性标识.

\subsection{IE 6-8 CSS3 媒体查询(Media
Query)}\label{ie-6-8-css3-ux5a92ux4f53ux67e5ux8be2media-query}

引入 \href{https://github.com/scottjehl/Respond}{Respond.js}.

\begin{lstlisting}[language=HTML]
<!--[if lt IE 9]>
<script src="respond.min.js"></script>
<![endif]-->
\end{lstlisting}

\section{跨域资源共享/CORS (Cross-origin resource
sharing)}\label{ux8de8ux57dfux8d44ux6e90ux5171ux4eabcors-cross-origin-resource-sharing}

\begin{itemize}
\tightlist
\item
  \href{http://www.alloyteam.com/2013/11/the-second-version-universal-solution-iframe-cross-domain-communication/}{TAT.Johnny,
  iframe跨域通信的通用解决方案, alloyteam}
\item
  \href{http://segmentfault.com/a/1190000003642057}{JasonKidd,「JavaScript」四种跨域方式详解,
  segmentfault}
\end{itemize}

\section{跨站请求伪造/CSRF (Cross-site request
forgery)}\label{ux8de8ux7ad9ux8bf7ux6c42ux4f2aux9020csrf-cross-site-request-forgery}

\section{HTML}\label{html}

\subsection{参数(input)在 form
之外}\label{ux53c2ux6570inputux5728-form-ux4e4bux5916}

input 在 form 之外时, 在 input 元素内添加 form 属性值为 form 的
ID\footnote{\href{http://www.dreamdealer.nl/articles/form_fields_outside_a_form.html}{PLACING
  FORM FIELDS OUTSIDE THE FORM TAG}}. 这样 input
仍然可以看做隶属于此表单, jQuery \lstinline!$('#formid').serialize();!
能够获取 form 之外的输入框值. 或者在提交 (\lstinline!submit!)
表单时会同样提交 \lstinline!outside! 这个值.

\begin{lstlisting}[language=HTML]
<form id="formid" method="get">

    <label>Name:</label>
    <input type="text" id="name" name="name">

    <label>Email:</label>
    <input type="email" id="email" name="email">
    <input type="submit" form="contact_form" value="send form" />
</form>
<input type="text" name="outside" form="formid">
\end{lstlisting}

注意: IE8 \lstinline!$('#formid').serialize();! 无法获取
\lstinline!outside! 值.

\section{模块化与组件化}\label{ux6a21ux5757ux5316ux4e0eux7ec4ux4ef6ux5316}

\begin{quote}
内嵌框架图(embedded)
\end{quote}

模块化, 强调内聚, 包含完整的业务逻辑, 可以方便业务的复用. 组件化,
强调复用, 重点在于接口的暴露, 和构件的概念类似.

\subsection{实现模块化}\label{ux5b9eux73b0ux6a21ux5757ux5316}

业务相关的特殊性都应包含在同一个模块内, 具体到前端,
这些特性包括与业务相关的接口、状态、路由等.

\subsection{实现组件化}\label{ux5b9eux73b0ux7ec4ux4ef6ux5316}

关键是如何暴露接口, 方便外部复用.

\section{权限管理}\label{ux6743ux9650ux7ba1ux7406}

\subsection{AngularJS
分角色登录}\label{angularjs-ux5206ux89d2ux8272ux767bux5f55}

不同角色/权限登录后所见菜单不一样. 方案如下:

\begin{enumerate}
\def\labelenumi{\arabic{enumi}.}
\tightlist
\item
  给不同的路由配置其角色/权限属性.
\item
  登录进入时, 记录角色/权限.
\item
  进入主页, 根据角色/权限构建菜单(view中包含全部菜单,非此角色菜单移除
  Dom).
\item
  点击菜单进入到对应路由时,
  根据判断路由的角色/权限属性是否和登录进入时记录的一样.
\end{enumerate}

此方案包括两部分的权限限制, 其一是将不必要的菜单移除 Dom,
但菜单对应的路由依然可用, 只是在页面上没有对应可操作的视图,
其二是路由和登录的角色/权限匹配. 以 AngularJS 为例,
对应每个步骤的代码如下.

\begin{enumerate}
\def\labelenumi{\arabic{enumi}.}
\tightlist
\item
  路由属性.
\end{enumerate}

\begin{lstlisting}
$stateProvider.state('Registration.Instructors', {
     url: "/Instructors",
     templateUrl: '/Scripts/App/Instructors/Templates/instructors.html',
     controller: 'InstructorController',
     data: { auth: "Admin"}
})
\end{lstlisting}

\begin{enumerate}
\def\labelenumi{\arabic{enumi}.}
\setcounter{enumi}{1}
\tightlist
\item
  登录用户权限/角色信息可记录到 \lstinline!rootScope! 中, 比如
  \lstinline!rootScope.adminType = "Admin"!.
\item
  菜单保留与移除. \lstinline!ng-if="adminType==='Admin'"!\footnote{\href{http://stackoverflow.com/questions/19177732/what-is-the-difference-between-ng-if-and-ng-show-ng-hide}{what
    is the difference between ng-if and ng-show/ng-hide}}.
\item
  路由和登录角色/权限匹配\footnote{\href{http://stackoverflow.com/questions/20978248/angularjs-conditional-routing-in-app-config}{angularjs:
    conditional routing in app.config}}.
\end{enumerate}

\begin{lstlisting}
app.run(function($rootScope){
  $rootScope.$on('$stateChangeStart', function(event, toState, toParams, fromState, fromParams){
    if ( toState.data.auth !== $rootScope.adminType ) {
        event.preventDefault();
        return false;
    }
  })
});
\end{lstlisting}

\subsection{Vue.js 权限管理}\label{vue.js-ux6743ux9650ux7ba1ux7406}

\begin{itemize}
\tightlist
\item
  \href{http://refined-x.com/2017/08/29/\%E5\%9F\%BA\%E4\%BA\%8EVue\%E5\%AE\%9E\%E7\%8E\%B0\%E5\%90\%8E\%E5\%8F\%B0\%E7\%B3\%BB\%E7\%BB\%9F\%E6\%9D\%83\%E9\%99\%90\%E6\%8E\%A7\%E5\%88\%B6/}{基于Vue实现后台系统权限控制}
\item
  \href{http://refined-x.com/2017/09/01/\%E7\%94\%A8addRoutes\%E5\%AE\%9E\%E7\%8E\%B0\%E5\%8A\%A8\%E6\%80\%81\%E8\%B7\%AF\%E7\%94\%B1/}{用addRoutes实现动态路由}
\item
  \href{https://juejin.im/post/591aa14f570c35006961acac}{手摸手,带你用vue撸后台
  系列二(登录权限篇)}
\item
  \href{https://www.zhihu.com/question/58991978}{Vue
  后台管理控制用户权限的解决方案?}
\item
  \href{https://cn.vuejs.org/v2/guide/custom-directive.html}{自定义指令}
\item
  \url{https://codepen.io/diemah77/pen/GZGxPK}
\end{itemize}

\section{npm包及私有库}\label{npmux5305ux53caux79c1ux6709ux5e93}

\subsection{npm 包编写}\label{npm-ux5305ux7f16ux5199}

npm 包通常会兼容不同的应用场景: nodejs、require.js 和浏览器.
所以会包含一段用于判断运行环境的代码, 如下.

\begin{lstlisting}
// from Vue.js
(function (global, factory) {
  typeof exports === 'object' && typeof module !== 'undefined' ? module.exports = factory() :
  typeof define === 'function' && define.amd ? define(factory) : (global.NPM_THING = factory());
}(this, (function () {
  'use strict';
  // code
  // return NPM_THING;
})));
\end{lstlisting}

\begin{enumerate}
\def\labelenumi{\arabic{enumi}.}
\tightlist
\item
  判断是否是 nodejs 环境:
  \lstinline"typeof exports === 'object' && typeof module !== 'undefined'".
\item
  判断是否是 require.js:
  \lstinline!typeof define === 'function' && define.amd!
\end{enumerate}

其中的 \lstinline!this! 指向全局变量, nodejs \lstinline!this! 为
\lstinline!global!, 浏览器中 \lstinline!this! 为 \lstinline!window!.

\subsection{私有库方案}\label{ux79c1ux6709ux5e93ux65b9ux6848}

为了避免重复造轮子, 提供编码效率, 同时又可以避免企业内部的业务逻辑暴露,
于是对私有库有需求. 期望, 如果私有库中有, 则从私有库中下载,
否则从公开的库中下载.

npm 的包都是公开的,
提供的\href{https://www.npmjs.com/enterprise}{企业私有化方案}是收费的.
开源方案有:

\begin{enumerate}
\def\labelenumi{\arabic{enumi}.}
\tightlist
\item
  cnpm: \url{https://github.com/cnpm/cnpmjs.org}
\item
  sinopia: \url{https://github.com/rlidwka/sinopia}
\end{enumerate}

两者的对比(\href{https://www.jianshu.com/p/659fb418c9e3}{企业私有 npm
服务器}):

\begin{longtable}[]{@{}lll@{}}
\toprule
-- & cnpm & sinopia\tabularnewline
\midrule
\endhead
系统支持 & 非windows & 全系统\tabularnewline
安装 & 复杂 & 简单\tabularnewline
配置 & 较多,适合个性化需求较多的 & 较少\tabularnewline
配置------修改默认镜像 & 不支持 & 支持\tabularnewline
存储 & mysql & 文件格式,直观\tabularnewline
服务托管 & 默认后台运行 & pm2, doker, forever\tabularnewline
文档资料 & 较多 & 较少\tabularnewline
\bottomrule
\end{longtable}

\section{前后端分离}\label{ux524dux540eux7aefux5206ux79bb}

内容简述: 简单描述前后端分离的历史状况, 明确前后端划分的原则:
后端面向数据, 前端面向用户, 在服务端引入 nodejs
可以解决的问题(应用的场景).

\subsection{历史}\label{ux5386ux53f2}

\begin{enumerate}
\def\labelenumi{\arabic{enumi}.}
\tightlist
\item
  前后端耦合. 例如, ASP.NET Webform 和 jsp 的标记语言的写法,
  每次请求由后端返回, 且后端的语言变量混在 HTML 标签中.
\item
  前后端半分离. 例如, ASP.NET MVC 和 Spring MVC 视图由后端控制, V (视图)
  由前端人员开发. 开发新的页面需要后端新建接口,
  编程语言通常在一个工程中, and so on.
\item
  前后端完全分离. 前后端通过接口联系. 前后端会有部分逻辑重合,
  比如用户输入的校验, 通常后端接口也会处理一次. 前端获取数据后渲染视图,
  SEO 困难.
\end{enumerate}

\subsection{目标/方法}\label{ux76eeux6807ux65b9ux6cd5}

\begin{enumerate}
\def\labelenumi{\arabic{enumi}.}
\tightlist
\item
  后端: 数据处理; 前端: 用户交互\footnote{\href{https://medium.com/@balint_sera/on-the-separation-of-front-end-and-backend-7a0809b42820}{Balint
    Sera, On the separation of front-end and backend, 2016-06-15}}.
\item
  前端向后扩展(服务端nodejs): 解决
  SEO、首屏优化、部分业务逻辑复用等问题; 前端向前扩展:
  实现跨终端(iOS和Android, H5, PC)代码复用.
\end{enumerate}

解决的问题:

\begin{enumerate}
\def\labelenumi{\arabic{enumi}.}
\tightlist
\item
  core: 优化交互体验, 提高编码效率.
\item
  SEO.
\item
  性能优化.
\item
  首屏优化.
\item
  代码复用(业务逻辑, 路由, 模板).
\end{enumerate}

\subsection{应用}\label{ux5e94ux7528}

\begin{enumerate}
\def\labelenumi{\arabic{enumi}.}
\tightlist
\item
  服务端框架: \href{https://github.com/koajs/koa}{Koa},
  \href{https://github.com/expressjs/express}{Express}.
\item
  同构框架支持: nuxt.js(可用 Koa 替换 Express).
\item
  路由用 history mode (Vue.js), 如果后端不配置, 直接进入页面无法访问.
  可复用模板, 直接访问时后端渲染, 路由访问时前端渲染\footnote{\href{http://2014.jsconf.cn/slides/herman-taobaoweb/\#/}{赫门,
    淘宝前后端分离实践, 2014}}.
\item
  服务端, 浏览器端及Native端都可应用的第三方库: axios, moment.js.
\end{enumerate}

\subsection{引入 nodejs
层的应用场景}\label{ux5f15ux5165-nodejs-ux5c42ux7684ux5e94ux7528ux573aux666f}

除了上面所述的性能和SEO等问题, 还可以作为中间件, 抹平同类型系统的差异,
构建统一的平台.

比如, 对于多个定制化的产品, 每个产品都对应有运营平台,
用于观测用户使用情况. 由于历史上造成的差异,
每个运营平台都需要重新构建一套运行于浏览器端的前端工程.
对于这种业务相似度较高的情况, 就可以在服务端引入 nodejs, 构建统一的平台,
抹平已有系统之前的差异(比如接口有不同的风格), 只需要实现一套 Web APP.
同时也方便了后期其他定制产品的扩展.

\subsection{扩展}\label{ux6269ux5c55}

\begin{enumerate}
\def\labelenumi{\arabic{enumi}.}
\tightlist
\item
  \href{https://juejin.im/post/598aabe96fb9a03c335a8dde}{美团点评点餐,
  美团点评点餐 Nuxt.js 实战, 2017-08-09}
\item
  \href{https://book.douban.com/subject/27183584/}{Jason Strimpel,
  Maxime Najim, 同构JavaScript应用开发, 2017}
\item
  \href{https://www.nczonline.net/blog/2013/10/07/node-js-and-the-new-web-front-end/}{Nicholas
  C. Zakas, Node.js and the new web front-end, 2013-10-07}
\end{enumerate}
