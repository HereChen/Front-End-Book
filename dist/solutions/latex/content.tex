\section{兼容性问题解决}\label{ux517cux5bb9ux6027ux95eeux9898ux89e3ux51b3}

\subsection{IE: 盒模型}\label{ie-ux76d2ux6a21ux578b}

IE 默认情况下长宽包含 \lstinline!padding! 和 \lstinline!border!,
和其他浏览器的长宽存在区别, 建议添加 \lstinline!border-sizing! 属性.
保持多个浏览器的一致性.

\begin{lstlisting}
box-sizing: border-box;
\end{lstlisting}

\subsection{IE 8: map}\label{ie-8-map}

IE8 不支持 JavaScript 原生 map 函数,
可在任意地方加入如下的代码片段\footnote{\href{http://stackoverflow.com/questions/7350912/is-the-javascript-map-function-supported-in-ie8}{Is
  the javascript .map() function supported in IE8?}}.

\begin{lstlisting}
(function (fn) {
    if (!fn.map) fn.map = function (f) {
        var r = [];
        for (var i = 0; i < this.length; i++)
            if (this[i] !== undefined) r[i] = f(this[i]);
        return r
    }
    if (!fn.filter) fn.filter = function (f) {
        var r = [];
        for (var i = 0; i < this.length; i++)
            if (this[i] !== undefined && f(this[i])) r[i] = this[i];
        return r
    }
})(Array.prototype);
\end{lstlisting}

或者用 jQuery 的 map 函数.

\begin{lstlisting}
// array.map(function( ) { });
jQuery.map(array, function( ) {
}
\end{lstlisting}

\subsection{IE 8: fontawesome
图标显示为方块}\label{ie-8-fontawesome-ux56feux6807ux663eux793aux4e3aux65b9ux5757}

对于修饰性不影响功能的图标, 可以做降级处理, 仅在非 IE 或者 IE9+
(条件注释\footnote{\href{https://msdn.microsoft.com/en-us/library/ms537512(v=vs.85).aspx}{About
  conditional comments}})情况下引入 fontawesome 图标库.
(谷歌搜索了一堆方案都没用, 最后应用这种方式来解决).

\begin{lstlisting}[language=HTML]
<!--[if (gt IE 8) | !IE]><!-->
<link rel="stylesheet" href="font-awesome.min.css">
<!--<![endif]-->
\end{lstlisting}

\subsection{IE 10+
浏览器定位}\label{ie-10-ux6d4fux89c8ux5668ux5b9aux4f4d}

IE 10+ 不支持条件注释, 因此需要其他方式定位这些浏览器. 如果只增加 CSS,
可采用以下方式定位\footnote{\href{http://stackoverflow.com/questions/9900311/how-do-i-target-only-internet-explorer-10-for-certain-situations-like-internet-e/14916454\#14916454}{How
  do I target only Internet Explorer 10 for certain situations like
  Internet Explorer-specific CSS or Internet Explorer-specific
  JavaScript code?}}.

\begin{lstlisting}
/*IE 9+, 以及 Chrome*/
@media screen and (min-width:0\0) {
}

/*IE 10*/
@media all and (-ms-high-contrast: none), (-ms-high-contrast: active) {
}

/*Edge*/
@supports (-ms-accelerator:true) {
}
\end{lstlisting}

另一种方式是 JavaScript 检测浏览器版本, 在 \lstinline!body!
标签为特定浏览器添加 \lstinline!class! 属性标识.

\subsection{IE 6-8 CSS3 媒体查询(Media
Query)}\label{ie-6-8-css3-ux5a92ux4f53ux67e5ux8be2media-query}

引入 \href{https://github.com/scottjehl/Respond}{Respond.js}.

\begin{lstlisting}[language=HTML]
<!--[if lt IE 9]>
<script src="respond.min.js"></script>
<![endif]-->
\end{lstlisting}

\section{HTML}\label{html}

\subsection{参数(input)在 form
之外}\label{ux53c2ux6570inputux5728-form-ux4e4bux5916}

input 在 form 之外时, 在 input 元素内添加 form 属性值为 form 的
ID\footnote{\href{http://www.dreamdealer.nl/articles/form_fields_outside_a_form.html}{PLACING
  FORM FIELDS OUTSIDE THE FORM TAG}}. 这样 input
仍然可以看做隶属于此表单, jQuery \lstinline!$('#formid').serialize();!
能够获取 form 之外的输入框值. 或者在提交 (\lstinline!submit!)
表单时会同样提交 \lstinline!outside! 这个值.

\begin{lstlisting}[language=HTML]
<form id="formid" method="get">

    <label>Name:</label>
    <input type="text" id="name" name="name">

    <label>Email:</label>
    <input type="email" id="email" name="email">
    <input type="submit" form="contact_form" value="send form" />
</form>
<input type="text" name="outside" form="formid">
\end{lstlisting}

注意: IE8 \lstinline!$('#formid').serialize();! 无法获取
\lstinline!outside! 值.

\section{SOLUTIONS/解决方案}\label{solutionsux89e3ux51b3ux65b9ux6848}

\section{系统方案}\label{ux7cfbux7edfux65b9ux6848}

\subsection{分角色登录}\label{ux5206ux89d2ux8272ux767bux5f55}

不同角色/权限登录后所见菜单不一样. 方案如下:

\begin{enumerate}
\def\labelenumi{\arabic{enumi}.}
\tightlist
\item
  给不同的路由配置其角色/权限属性.
\item
  登录进入时, 记录角色/权限.
\item
  进入主页, 根据角色/权限构建菜单(view中包含全部菜单,非此角色菜单移除
  Dom).
\item
  点击菜单进入到对应路由时,
  根据判断路由的角色/权限属性是否和登录进入时记录的一样.
\end{enumerate}

此方案包括两部分的权限限制, 其一是将不必要的菜单移除 Dom,
但菜单对应的路由依然可用, 只是在页面上没有对应可操作的视图,
其二是路由和登录的角色/权限匹配. 以 AngularJS 为例,
对应每个步骤的代码如下.

\begin{enumerate}
\def\labelenumi{\arabic{enumi}.}
\tightlist
\item
  路由属性.
\end{enumerate}

\begin{lstlisting}
$stateProvider.state('Registration.Instructors', {
     url: "/Instructors",
     templateUrl: '/Scripts/App/Instructors/Templates/instructors.html',
     controller: 'InstructorController',
     data: { auth: "Admin"}
})
\end{lstlisting}

\begin{enumerate}
\def\labelenumi{\arabic{enumi}.}
\setcounter{enumi}{1}
\tightlist
\item
  登录用户权限/角色信息可记录到 \lstinline!rootScope! 中, 比如
  \lstinline!rootScope.adminType = "Admin"!.
\item
  菜单保留与移除. \lstinline!ng-if="adminType==='Admin'"!\footnote{\href{http://stackoverflow.com/questions/19177732/what-is-the-difference-between-ng-if-and-ng-show-ng-hide}{what
    is the difference between ng-if and ng-show/ng-hide}}.
\item
  路由和登录角色/权限匹配\footnote{\href{http://stackoverflow.com/questions/20978248/angularjs-conditional-routing-in-app-config}{angularjs:
    conditional routing in app.config}}.
\end{enumerate}

\begin{lstlisting}
app.run(function($rootScope){
  $rootScope.$on('$stateChangeStart', function(event, toState, toParams, fromState, fromParams){
    if ( toState.data.auth !== $rootScope.adminType ) {
        event.preventDefault();
        return false;
    }
  })
});
\end{lstlisting}

