\section{介绍}\label{ux4ecbux7ecd}

\begin{enumerate}
\def\labelenumi{\arabic{enumi}.}
\tightlist
\item
  \href{https://github.com/jquery/jquery}{jQuery}: Dom 工具.
\item
  \href{https://github.com/lodash/lodash}{lodash}: A modern JavaScript
  utility library delivering modularity, performance, \& extras.
\end{enumerate}

\section{Vue.js}\label{vue.js}

\href{https://github.com/vuejs/vue}{Vue.js} 开发简单直观,
简单实用的东西通常寿命会比较长.

\subsection{周边配套}\label{ux5468ux8fb9ux914dux5957}

\begin{enumerate}
\def\labelenumi{\arabic{enumi}.}
\tightlist
\item
  开发小程序:
  \href{https://github.com/Meituan-Dianping/mpvue}{Meituan-Dianping/mpvue}
\item
  开发原生APP: \href{https://weex.apache.org/}{weex}
\end{enumerate}

\subsection{Tips}\label{tips}

\subsubsection{ES6}\label{es6}

以下几个 ES6 功能应用于 Vue.js 将获得不错的收益\footnote{\href{https://vuejsdevelopers.com/2018/01/22/vue-js-javascript-es6/}{ANTHONY
  GORE, 4 Essential ES2015 Features For Vue.js Development, 2018-01-22}},
特别是对于无需构建工具的情况.

\begin{enumerate}
\def\labelenumi{\arabic{enumi}.}
\item
  箭头函数: 让 this 始终指向到 Vue 实例上.
\item
  模板字符串: 应用于 Vue 行内模板, 可以方便换行, 无需用加号链接.
  也可以应用于变量套入到字符串中.

\begin{lstlisting}
Vue.component({
  template: `<div>
              <h1></h1>
              <p></p>
            </div>`
  data: {
    time: `time: ${Date.now()}`
  }
});
\end{lstlisting}
\item
  模块(Modules): 应用于声明式的组件 \lstinline!Vue.component!,
  甚至不需要 webpack 的支持.

\begin{lstlisting}
import component1 from './component1.js';
Vue.component('component1', component1);
\end{lstlisting}
\item
  解构赋值: 可应用于只获取需要的值, 减少不必要的赋值, 比如只获取 Vuex
  中的 commit 而不需要 store.

\begin{lstlisting}
actions: {
  increment ({ commit }) {
    commit(...);
  }
}
\end{lstlisting}
\item
  扩展运算符: 数组和对象等批量导出, 而不需要用循环语句. 比如,
  将路由根据功能划分为多个文件, 再用扩展展运算符在 index 中合在一起.
\end{enumerate}

\subsubsection{组件重新渲染}\label{ux7ec4ux4ef6ux91cdux65b0ux6e32ux67d3}

通过设置 \lstinline!v-if! 实现, 从 Dom 中剔除再加入.

\begin{lstlisting}[language=HTML]
<demo-component v-if="ifShow"></demo-component>
\end{lstlisting}

\subsubsection{绑定数据后添加属性视图未重新渲染}\label{ux7ed1ux5b9aux6570ux636eux540eux6dfbux52a0ux5c5eux6027ux89c6ux56feux672aux91cdux65b0ux6e32ux67d3}

如果存在异步请求, 在数据上添加属性的情况, 需要先预处理好获取的数据,
然后在将其赋值到 data 中变量. 数据绑定后, 再添加属性, 不会触发界面渲染.

\begin{lstlisting}
API.getSomething().then(res => {
  // 1. 先添加属性
  // handle 表示对数据的处理, 包括对象中属性的添加
  const handledRes = handle(res);
  // 2. 然后绑定到 data 中的变量
  this.varInDate = handledRes;
});
\end{lstlisting}

\subsubsection[全局引入 SCSS 变量文件]{\texorpdfstring{全局引入 SCSS
变量文件\footnote{\url{https://www.reddit.com/r/vuejs/comments/7o663j/sassscss_in_vue_where_to_store_variables/?st=JC9T45PB\&sh=4f87ec9d}}}{全局引入 SCSS 变量文件}}\label{ux5168ux5c40ux5f15ux5165-scss-ux53d8ux91cfux6587ux4ef6vueglobalimportvariablesfile}

场景: 将常用的变量存储到 \lstinline!vars.scss!,
应用变量时需要在每个需要的地方 \lstinline!import!.

\begin{enumerate}
\def\labelenumi{\arabic{enumi}.}
\tightlist
\item
  \lstinline!npm install sass-resources-loader --save-dev!
\item
  更改 \lstinline!build/webpack.base.conf.js!, 适用于 vue-cli.
\end{enumerate}

\begin{lstlisting}
{
    test: /\.vue$/,
    loader: 'vue-loader',
    options: {
        loaders: {
            sass: ['vue-style-loader', 'css-loader', {
                loader: 'sass-loader',
                options: {
                    indentedSyntax: true
                }
            }, {
                loader: 'sass-resources-loader',
                options: {
                    resources: path.resolve(__dirname, "./styles/vars.scss")
                }
            }],
            scss: ['vue-style-loader', 'css-loader', 'sass-loader', {
                loader: 'sass-resources-loader',
                options: {
                    resources: path.resolve(__dirname, "./styles/vars.scss")
                }
            }]
        }
        // other vue-loader options go here
    }
}
\end{lstlisting}

\subsection{Compatible}\label{compatible}

\subsubsection{\texorpdfstring{IE
\texttt{vuex\ requires\ a\ promise\ polyfill\ in\ this\ browser}}{IE vuex requires a promise polyfill in this browser}}\label{ie-vuex-requires-a-promise-polyfill-in-this-browser}

\begin{lstlisting}[language=bash]
npm install --save-dev babel-polyfill
\end{lstlisting}

\begin{lstlisting}
// build/webpack.base.conf.js
entry: {
  app: [
    'babel-polyfill',
    './src/main.js'
  ]
}
\end{lstlisting}

\href{https://github.com/vuejs-templates/webpack/issues/474}{vuex
requires a promise polyfill in this browser}

\section{React}\label{react}

\begin{lstlisting}[language=bash]
npm install -g create-react-app
create-react-app my-app
\end{lstlisting}

\href{https://reactjs.org/}{React} 与 Vue
类似,主要专注于视图层,结合其他的库实现扩展,比如路由、数据状态维护等。

\subsection{常用依赖}\label{ux5e38ux7528ux4f9dux8d56}

\begin{enumerate}
\def\labelenumi{\arabic{enumi}.}
\tightlist
\item
  路由: \href{https://github.com/ReactTraining/react-router}{React
  Router}
\item
  类型检查: \href{https://github.com/facebook/prop-types}{prop-types}
\item
  数据管理: \href{https://github.com/reactjs/react-redux}{React Redux}
\end{enumerate}

\subsubsection{UI}\label{ui}

\begin{itemize}
\tightlist
\item
  \href{https://eleme.github.io/element-react/}{element-react}
\item
  \href{https://ant.design/}{ant-design}
\end{itemize}

\subsubsection{优化}\label{ux4f18ux5316}

\begin{itemize}
\tightlist
\item
  \href{https://github.com/reactjs/reselect}{reselector}
\item
  \href{https://github.com/facebook/immutable-js}{immutable.js}
\item
  \href{https://github.com/rtfeldman/seamless-immutable}{seamless-immutable.js}
\end{itemize}

\subsection{文件组织}\label{ux6587ux4ef6ux7ec4ux7ec7}

\subsection{React 与 Redux}\label{react-ux4e0e-redux}

reducer 和 action 分文件夹放置,reducer 通过 combineReducers
合并,然后再通过 createStore 创建 store。store 通过 Provider 放置在 App
的顶层元素,并分发给所有子组件。对每个具体的功能,分别按需引入需要的
state (reducer 中定义)和 action,并用 connect 将此两项和视图层(react
实现)连接起来,实现完整的组件。由此,实现了数据、视图、事件之间的分离,通过
connect 实现连接。

\subsection{扩展}\label{ux6269ux5c55}

\begin{enumerate}
\def\labelenumi{\arabic{enumi}.}
\tightlist
\item
  \href{https://github.com/erikras/ducks-modular-redux}{GitHub,
  erikras/ducks-modular-redux}
\end{enumerate}

\section{React Native}\label{react-native}

\begin{enumerate}
\def\labelenumi{\arabic{enumi}.}
\tightlist
\item
  主页: \url{https://facebook.github.io/react-native}
\item
  GitHub: \url{https://github.com/facebook/react-native}
\item
  示例项目:
  \href{https://github.com/jiwonbest/amazing-react-projects}{amazing-react-projects}
\item
  Demo Project:
  \href{https://github.com/HereChen/template/tree/master/react-native}{react-native}
\end{enumerate}

\subsection{环境配置}\label{ux73afux5883ux914dux7f6e}

\subsubsection{系统环境}\label{ux7cfbux7edfux73afux5883}

\begin{enumerate}
\def\labelenumi{\arabic{enumi}.}
\tightlist
\item
  安装 \href{https://nodejs.org}{nodejs}.
\item
  \lstinline!npm install -g react-native-cli!.
\end{enumerate}

\paragraph{Android}\label{android}

\begin{enumerate}
\def\labelenumi{\arabic{enumi}.}
\tightlist
\item
  JDK (并配置环境变量)
\item
  安装 Android Studio \url{http://www.android-studio.org}
\item
  通过 SDK Manager 下载 SDK, 并配置环境变量.
\end{enumerate}

\begin{lstlisting}[language=bash]
REM set var
set ANDROID_HOME=C:\Users\chenl\AppData\Local\Android\Sdk

REM set Android home path
setx /m ANDROID_HOME "%ANDROID_HOME%"

REM set path
setx /m path "%path%;%ANDROID_HOME%\tools;%ANDROID_HOME%\platform-tools;"
\end{lstlisting}

\paragraph{iOS}\label{ios}

\begin{enumerate}
\def\labelenumi{\arabic{enumi}.}
\item
  App Store 安装 XCode.
\item
  其他工具安装

\begin{lstlisting}[language=bash]
brew install node
brew install watchman
npm install -g react-native-cli
\end{lstlisting}
\end{enumerate}

\subsubsection{编辑器}\label{ux7f16ux8f91ux5668}

\begin{enumerate}
\def\labelenumi{\arabic{enumi}.}
\tightlist
\item
  Visual Studio Code. 安装扩展 \lstinline!React Native Tools! 用于调试.
\item
  Atom. 安装\href{https://atom.io/packages/nuclide}{nuclide}.
\end{enumerate}

\subsubsection{参考}\label{ux53c2ux8003}

\begin{enumerate}
\def\labelenumi{\arabic{enumi}.}
\tightlist
\item
  \url{https://facebook.github.io/react-native/docs/getting-started.html}
\end{enumerate}

\subsection{基本命令}\label{ux57faux672cux547dux4ee4}

\begin{enumerate}
\def\labelenumi{\arabic{enumi}.}
\tightlist
\item
  新建工程: \lstinline!react-native init demo-project!.
\item
  Android 运行: \lstinline!react-native run-android!.
\item
  iOS 运行: \lstinline!react-native run-ios!.
\end{enumerate}

新建工程后首先 \lstinline!npm install! 安装依赖. 示例项目 python 和
node-gyp-bin 相关错误可以尝试先执行 \lstinline!yarn add node-sass! 或者
\lstinline!npm install -f node-sass!
(\url{https://github.com/sass/node-sass/issues/1980}).

\subsection{打包}\label{ux6253ux5305}

\subsubsection{Android 打包}\label{android-ux6253ux5305}

\paragraph{生成签名密钥}\label{ux751fux6210ux7b7eux540dux5bc6ux94a5}

\begin{lstlisting}[language=bash]
$ keytool -genkey -v -keystore my-release-key.keystore -alias my-key-alias -keyalg RSA -keysize 2048 -validity 10000
Enter keystore password:
Keystore password is too short - must be at least 6 characters
Enter keystore password: chenlei
Re-enter new password: chenlei
What is your first and last name?
  [Unknown]:  HereChen
What is the name of your organizational unit?
  [Unknown]:  HereChen
What is the name of your organization?
  [Unknown]:  HereChen
What is the name of your City or Locality?
  [Unknown]:  Chengdu
What is the name of your State or Province?
  [Unknown]:  Sichuan
What is the two-letter country code for this unit?
  [Unknown]:  51
Is CN=HereChen, OU=HereChen, O=HereChen, L=Chengdu, ST=Sichuan, C=51 correct?
  [no]:  yes

Generating 2,048 bit RSA key pair and self-signed certificate (SHA256withRSA) with a validity of 10,000 days
        for: CN=HereChen, OU=HereChen, O=HereChen, L=Chengdu, ST=Sichuan, C=51
Enter key password for <my-key-alias>
        (RETURN if same as keystore password):
[Storing my-release-key.keystore]
\end{lstlisting}

\paragraph{gradle设置}\label{gradleux8bbeux7f6e}

\begin{enumerate}
\def\labelenumi{\arabic{enumi}.}
\item
  \lstinline!my-release-key.keystore! 文件放到工程
  \lstinline!android/app! 文件夹下.
\item
  编辑 \lstinline!android/app/gradle.properties!, 添加如下信息.

\begin{lstlisting}
MYAPP_RELEASE_STORE_FILE=my-release-key.keystore
MYAPP_RELEASE_KEY_ALIAS=my-key-alias
MYAPP_RELEASE_STORE_PASSWORD=chenlei
MYAPP_RELEASE_KEY_PASSWORD=chenlei
\end{lstlisting}
\item
  编辑 \lstinline!android/app/build.gradle!, 添加如下信息.

\begin{lstlisting}
...
android {
    ...
    defaultConfig { ... }
    signingConfigs {
        release {
            storeFile file(MYAPP_RELEASE_STORE_FILE)
            storePassword MYAPP_RELEASE_STORE_PASSWORD
            keyAlias MYAPP_RELEASE_KEY_ALIAS
            keyPassword MYAPP_RELEASE_KEY_PASSWORD
        }
    }
    buildTypes {
        release {
            ...
            signingConfig signingConfigs.release
        }
    }
}
...
\end{lstlisting}
\end{enumerate}

\paragraph{生成 apk}\label{ux751fux6210-apk}

\begin{lstlisting}[language=bash]
cd android && ./gradlew assembleRelease
\end{lstlisting}

打包后在 \lstinline!android/app/build/outputs/apk/app-release.apk!.

\paragraph{安装 apk 方式}\label{ux5b89ux88c5-apk-ux65b9ux5f0f}

\begin{enumerate}
\def\labelenumi{\arabic{enumi}.}
\tightlist
\item
  Genymotion 可以拖拽 apk 进行安装.
\item
  \lstinline!adb install app-release.apk! 安装.
\end{enumerate}

如果报签名错误, 可先卸载之前的 debug 版本.

\subsubsection{iOS 打包}\label{ios-ux6253ux5305}

iOS 版本编译需要在 Mac 上进行.

\paragraph{签名}\label{ux7b7eux540d}

没有证书\ldots{}.

\paragraph{生成 ipa}\label{ux751fux6210-ipa}

以下流程以 Xcode 9 为例.

\begin{enumerate}
\def\labelenumi{\arabic{enumi}.}
\tightlist
\item
  打开工程: Xcode 打开 \lstinline!ios! 文件夹下 \lstinline!*.xcodeproj!
  文件(工程).
\item
  选择编译机型: Xcode 虚拟机选择栏中选择 \lstinline!Generic iOS Device!.
\item
  编译设置: Xcode -\textgreater{} Product -\textgreater{} Scheme
  -\textgreater{} Edit Scheme -\textgreater{} Run -\textgreater{} Info
  -\textgreater{} Build Configuration 选择 Rlease
\item
  JS 改为离线(打包进APP)???
\end{enumerate}

TODO: 命令行打包

\subsubsection{参考}\label{ux53c2ux8003-1}

\begin{enumerate}
\def\labelenumi{\arabic{enumi}.}
\tightlist
\item
  \href{https://facebook.github.io/react-native/docs/signed-apk-android.html}{Generating
  Signed APK, Facebook Open Source}
\item
  \href{https://reactnative.cn/docs/0.51/signed-apk-android.html}{打包APK,
  React Native中文网}
\item
  \href{https://www.jianshu.com/p/1380d4c8b596}{ReactNative之Android打包APK方法(趟坑过程),
  ZPengs, 2017.02.09, 简书}
\end{enumerate}

\subsection{入口文件更改}\label{ux5165ux53e3ux6587ux4ef6ux66f4ux6539}

\begin{quote}
从0.49开始, 只有一个入口, 不区分 ios 和 android.
\url{https://github.com/facebook/react-native/releases/tag/v0.49.0}
\end{quote}

React Native CLI 新建的工程, 默认入口是 \lstinline!index.js!. 在
\lstinline!android\app\build.gradle! 中更改入口.

\begin{lstlisting}
project.ext.react = [
    entryFile: "index.android.js"
]
\end{lstlisting}

对应更改
\lstinline!android\app\src\main\java\com\**\MainApplication.java!.

\begin{lstlisting}[language=Java]
protected String getJSMainModuleName() {
  return "index.android";
}
\end{lstlisting}

\subsection{工具/依赖(dependencies)}\label{ux5de5ux5177ux4f9dux8d56dependencies}

\subsubsection{导航}\label{ux5bfcux822a}

\begin{quote}
\url{https://facebook.github.io/react-native/docs/navigation.html}
\end{quote}

\begin{enumerate}
\def\labelenumi{\arabic{enumi}.}
\tightlist
\item
  \href{https://github.com/react-navigation/react-navigation}{react-navigation}
  提供了常用的导航方式(Stack, Tab, Drawer), 推荐.
\item
  \href{https://facebook.github.io/react-native/docs/navigatorios.html}{NavigatorIOS}
  为内建的导航, 仅在 IOS 上可用.
\end{enumerate}

\subsubsection{UI}\label{ui-1}

尚未找到两端(Web, Native)完整好用的 UI, 若后端采用 ant-design 可用
ant-design-mobile.

\begin{enumerate}
\def\labelenumi{\arabic{enumi}.}
\tightlist
\item
  \href{https://github.com/ant-design/ant-design-mobile}{ant-design-mobile}
  每个组件是否支持 Native 有说明.
\item
  \href{https://github.com/react-native-training/react-native-elements}{react-native-elements}
\item
  \href{https://github.com/GeekyAnts/NativeBase}{NativeBase}
\end{enumerate}

\subsubsection{HTTP 请求}\label{http-ux8bf7ux6c42}

\begin{quote}
\url{https://facebook.github.io/react-native/docs/network.html}
\end{quote}

\begin{enumerate}
\def\labelenumi{\arabic{enumi}.}
\tightlist
\item
  \href{https://developer.mozilla.org/en-US/docs/Web/API/Fetch_API}{fetch}
  为内建接口.
\item
  \href{https://github.com/axios/axios}{\textbf{axios}}
  为使用校广泛的第三方请求库, 推荐使用.
\end{enumerate}

\subsection{调试}\label{ux8c03ux8bd5}

\begin{quote}
\url{https://facebook.github.io/react-native/docs/debugging.html}
\end{quote}

根据提示, 可以菜单按钮选择重新加载或热加载. Android 可摇晃手机显示菜单.

\subsubsection{虚拟机}\label{ux865aux62dfux673a}

\begin{enumerate}
\def\labelenumi{\arabic{enumi}.}
\tightlist
\item
  \href{https://www.genymotion.com/download/}{Genymotion}, 需要先注册,
  然后选择 for personal 使用. 如果系统开启了 Hyper-V, 需要先关闭.
\item
  Android Studio 内建虚拟机, 同样需要关闭 Hyper-V.
\item
  \href{https://www.visualstudio.com/vs/msft-android-emulator/}{Visual
  Studio Emulator for Android} 需要开启 Hyper-V.
\end{enumerate}

\subsubsection{调试工具: Chrome}\label{ux8c03ux8bd5ux5de5ux5177-chrome}

\begin{enumerate}
\def\labelenumi{\arabic{enumi}.}
\tightlist
\item
  \lstinline!Remote JS Debugging! 开启JS调试.
\item
  浏览器端进去 \lstinline!http://localhost:8081/debugger-ui/!,
  并开启开发工具.
\item
  可在 Sources 中设置断点或者代码中写入 \lstinline!debugger!.
\end{enumerate}

\subsubsection{调试工具: VSCode}\label{ux8c03ux8bd5ux5de5ux5177-vscode}

\begin{enumerate}
\def\labelenumi{\arabic{enumi}.}
\tightlist
\item
  安装扩展: React Native Tools.
\item
  F5 生成 launch.json 文件.
\item
  进入调试菜单(Ctrl + Shift + D), 选择 Debug Android.
\item
  设置断点或者写入 \lstinline!debugger! 开始调试, 在 output 栏输出.
\end{enumerate}

\subsubsection{HTTP
调试问题备注}\label{http-ux8c03ux8bd5ux95eeux9898ux5907ux6ce8}

应用 Fiddler 调试 HTTP, 模拟器设置了代理后, APP 无法热加载 JS bundle.
目前只有用 Chrome 或者断点的方式来调试.

\subsection{工程结构}\label{ux5de5ux7a0bux7ed3ux6784}

\subsubsection{结构}\label{ux7ed3ux6784}

\begin{lstlisting}
android/         # Android 工程
ios/             # IOS 工程
src/             # 开发前端资源
  -- assets/     # 静态资源
  -- components/ # 组件
  -- api/        # 接口
  -- route/      # 导航(路由)
  -- config/     # 常量配置
  -- pages/      # 页面/功能
  -- utils/      # 常用工具
  -- reducers 相关
  -- index.js    # APP 入口
index.js         # 入口文件
\end{lstlisting}

\subsubsection{参考}\label{ux53c2ux8003-2}

\begin{enumerate}
\def\labelenumi{\arabic{enumi}.}
\tightlist
\item
  \href{https://medium.com/the-react-native-log/organizing-a-react-native-project-9514dfadaa0}{Organizing
  a React Native Project}
\item
  \href{https://hackernoon.com/manage-react-native-project-folder-structure-and-simplify-the-code-c98da77ef792}{React
  native project setup --- a better folder structure}
\end{enumerate}

\subsection{Tips}\label{tips-1}

\begin{enumerate}
\def\labelenumi{\arabic{enumi}.}
\tightlist
\item
  Android 查看当前的 Android 设备 \lstinline!adb devices!.
\item
  Android 虚拟机: Ctrl + M 打开菜单 (Android
  Studio自带虚拟机没有菜单和摇晃手机, 可以这种方式打开菜单).
\item
  iPhone 虚拟机啊重新加载资源: command + R.
\end{enumerate}

\subsection{问题及解决}\label{ux95eeux9898ux53caux89e3ux51b3}

\begin{enumerate}
\def\labelenumi{\arabic{enumi}.}
\tightlist
\item
  VSCode Debug 无法加载的情况, 首先重启 VSCode 再启动项目.
\item
  添加\lstinline!antd-mobile!后报错, 无法解析 \lstinline!react-dom!,
  依赖中加入\lstinline!react-dom!并安装即可.
\item
  集成\lstinline!react-native-navigation!需要注意Android SDK版本,
  版本过低可能出现编译错误(\lstinline!Error:Error retrieving parent for item: No resource found!).
\end{enumerate}

\subsection{原理}\label{ux539fux7406}

\begin{enumerate}
\def\labelenumi{\arabic{enumi}.}
\tightlist
\item
  React
  Native将代码由JSX转化为JS组件,启动过程中利用instantiateReactComponent将ReactElement转化为复合组件ReactCompositeComponent与元组件ReactNativeBaseComponent,利用
  ReactReconciler对他们进行渲染\footnote{\href{https://github.com/guoxiaoxing/react-native/blob/master/doc/ReactNative\%E6\%BA\%90\%E7\%A0\%81\%E7\%AF\%87/4ReactNative\%E6\%BA\%90\%E7\%A0\%81\%E7\%AF\%87\%EF\%BC\%9A\%E6\%B8\%B2\%E6\%9F\%93\%E5\%8E\%9F\%E7\%90\%86.md}{ReactNative源码篇:渲染原理}}。
\item
  UIManager.js利用C++层的Instance.cpp将UI信息传递给UIManagerModule.java,并利用UIManagerModule.java构建UI\footnote{\href{https://github.com/guoxiaoxing/react-native/blob/master/doc/ReactNative\%E6\%BA\%90\%E7\%A0\%81\%E7\%AF\%87/4ReactNative\%E6\%BA\%90\%E7\%A0\%81\%E7\%AF\%87\%EF\%BC\%9A\%E6\%B8\%B2\%E6\%9F\%93\%E5\%8E\%9F\%E7\%90\%86.md}{ReactNative源码篇:渲染原理}}。
\item
  UIManagerModule.java接收到UI信息后,将UI的操作封装成对应的Action,放在队列中等待执行。各种UI的操作,例如创建、销毁、更新等便在队列里完成,UI最终
  得以渲染在屏幕上\footnote{\href{https://github.com/guoxiaoxing/react-native/blob/master/doc/ReactNative\%E6\%BA\%90\%E7\%A0\%81\%E7\%AF\%87/4ReactNative\%E6\%BA\%90\%E7\%A0\%81\%E7\%AF\%87\%EF\%BC\%9A\%E6\%B8\%B2\%E6\%9F\%93\%E5\%8E\%9F\%E7\%90\%86.md}{ReactNative源码篇:渲染原理}}。
\end{enumerate}

\section{Weex}\label{weex}

\begin{enumerate}
\def\labelenumi{\arabic{enumi}.}
\tightlist
\item
  主页: \url{http://weex.apache.org}
\item
  GitHub: \url{https://github.com/apache/incubator-weex/}
\end{enumerate}

问题: 入口在哪儿?

\textbf{案例}

\begin{enumerate}
\def\labelenumi{\arabic{enumi}.}
\tightlist
\item
  \href{https://github.com/zwwill/yanxuan-weex-demo}{网易严选}
\item
  \href{https://mp.weixin.qq.com/s/dowOE_QpZrtV5GH9EAgyHg}{点我达骑手Weex最佳实践}
\item
  \href{https://github.com/weexteam/weex-hackernews}{weexteam/weex-hackernews}
\end{enumerate}

\subsection{搭建开发环境}\label{ux642dux5efaux5f00ux53d1ux73afux5883}

\begin{lstlisting}
npm install -g weex-toolkit
\end{lstlisting}

\subsection{Demo}\label{demo}

\textbf{web}

\begin{lstlisting}
weex create weex
cd weex
npm install
npm run dev & npm run serve
\end{lstlisting}

\textbf{命令}

\begin{quote}
https://github.com/weexteam/weex-pack
\end{quote}

\begin{lstlisting}[language=bash]
# debug
weex debug

# add platform
weex platform add android
weex platform add ios

# run
weex run web
weex run android
weex run ios

# build
weex build web
\end{lstlisting}

\subsection{问题及解决}\label{ux95eeux9898ux53caux89e3ux51b3-1}

\begin{enumerate}
\def\labelenumi{\arabic{enumi}.}
\tightlist
\item
  \lstinline!https://maven.google.com/! 链接不上,
  更改\lstinline!\platforms\android\build.gradle!文件, 换成
  \lstinline!https://dl.google.com/dl/android/maven2/!。
\item
  \lstinline!adb: failed to stat app/build/outputs/apk/playground.apk: No such file or directory!,
  替换 \lstinline!platforms/android/app/build.gradle! 文件中的
  \lstinline!weex-app.apk! 为 \lstinline!playground.apk!.
\item
  \lstinline!weex debug! 报错可先安装
  \lstinline!npm install -g weex-devtool!.
\end{enumerate}

\section{React Native vs Weex}\label{react-native-vs-weex}

\subsection{对比表格}\label{ux5bf9ux6bd4ux8868ux683c}

\begin{longtable}[]{@{}lll@{}}
\toprule
\begin{minipage}[b]{0.30\columnwidth}\raggedright\strut
属性\strut
\end{minipage} & \begin{minipage}[b]{0.30\columnwidth}\raggedright\strut
\href{https://github.com/facebook/react-native}{React Native}\strut
\end{minipage} & \begin{minipage}[b]{0.30\columnwidth}\raggedright\strut
\href{https://github.com/apache/incubator-weex/}{Weex}\strut
\end{minipage}\tabularnewline
\midrule
\endhead
\begin{minipage}[t]{0.30\columnwidth}\raggedright\strut
开源时间\strut
\end{minipage} & \begin{minipage}[t]{0.30\columnwidth}\raggedright\strut
2015/03\strut
\end{minipage} & \begin{minipage}[t]{0.30\columnwidth}\raggedright\strut
2016/06\strut
\end{minipage}\tabularnewline
\begin{minipage}[t]{0.30\columnwidth}\raggedright\strut
开源企业\strut
\end{minipage} & \begin{minipage}[t]{0.30\columnwidth}\raggedright\strut
Facebook\strut
\end{minipage} & \begin{minipage}[t]{0.30\columnwidth}\raggedright\strut
Alibaba\strut
\end{minipage}\tabularnewline
\begin{minipage}[t]{0.30\columnwidth}\raggedright\strut
协议\strut
\end{minipage} & \begin{minipage}[t]{0.30\columnwidth}\raggedright\strut
BSD 3-clause\strut
\end{minipage} & \begin{minipage}[t]{0.30\columnwidth}\raggedright\strut
Apache License 2.0\strut
\end{minipage}\tabularnewline
\begin{minipage}[t]{0.30\columnwidth}\raggedright\strut
主页标语\strut
\end{minipage} & \begin{minipage}[t]{0.30\columnwidth}\raggedright\strut
Build native mobile apps using JavaScript and React\strut
\end{minipage} & \begin{minipage}[t]{0.30\columnwidth}\raggedright\strut
A framework for building Mobile cross-paltform UIs\strut
\end{minipage}\tabularnewline
\begin{minipage}[t]{0.30\columnwidth}\raggedright\strut
核心理念\strut
\end{minipage} & \begin{minipage}[t]{0.30\columnwidth}\raggedright\strut
Learn Once, Write Anywhere\strut
\end{minipage} & \begin{minipage}[t]{0.30\columnwidth}\raggedright\strut
Write Once, Run Everywhere\strut
\end{minipage}\tabularnewline
\begin{minipage}[t]{0.30\columnwidth}\raggedright\strut
前端框架\strut
\end{minipage} & \begin{minipage}[t]{0.30\columnwidth}\raggedright\strut
React\strut
\end{minipage} & \begin{minipage}[t]{0.30\columnwidth}\raggedright\strut
Vue.js\strut
\end{minipage}\tabularnewline
\begin{minipage}[t]{0.30\columnwidth}\raggedright\strut
JS Engine\strut
\end{minipage} & \begin{minipage}[t]{0.30\columnwidth}\raggedright\strut
JavaScriptCore(iOS/Android)\strut
\end{minipage} & \begin{minipage}[t]{0.30\columnwidth}\raggedright\strut
JavaScriptCore(iOS) /v8(Android)\strut
\end{minipage}\tabularnewline
\begin{minipage}[t]{0.30\columnwidth}\raggedright\strut
三端开发\strut
\end{minipage} & \begin{minipage}[t]{0.30\columnwidth}\raggedright\strut
部分组件需要区分平台开发\strut
\end{minipage} & \begin{minipage}[t]{0.30\columnwidth}\raggedright\strut
强调三端统一\strut
\end{minipage}\tabularnewline
\begin{minipage}[t]{0.30\columnwidth}\raggedright\strut
代码写法\strut
\end{minipage} & \begin{minipage}[t]{0.30\columnwidth}\raggedright\strut
JSX(JavaScript + XML)\strut
\end{minipage} & \begin{minipage}[t]{0.30\columnwidth}\raggedright\strut
Web 写法\strut
\end{minipage}\tabularnewline
\begin{minipage}[t]{0.30\columnwidth}\raggedright\strut
调试\strut
\end{minipage} & \begin{minipage}[t]{0.30\columnwidth}\raggedright\strut
虚拟机\strut
\end{minipage} & \begin{minipage}[t]{0.30\columnwidth}\raggedright\strut
可用 Chrome 查看效果\strut
\end{minipage}\tabularnewline
\begin{minipage}[t]{0.30\columnwidth}\raggedright\strut
社区支持\strut
\end{minipage} & \begin{minipage}[t]{0.30\columnwidth}\raggedright\strut
社区活跃, 有多个流行产品的实践\strut
\end{minipage} & \begin{minipage}[t]{0.30\columnwidth}\raggedright\strut
目前, 开发者主要在国内, 没有太多的实践案例\strut
\end{minipage}\tabularnewline
\begin{minipage}[t]{0.30\columnwidth}\raggedright\strut
优势\strut
\end{minipage} & \begin{minipage}[t]{0.30\columnwidth}\raggedright\strut
生态好, 第三方依赖多, 有可借鉴的经验\strut
\end{minipage} & \begin{minipage}[t]{0.30\columnwidth}\raggedright\strut
基于 Vue.js, 上手快, 能更好的保证三端一致\strut
\end{minipage}\tabularnewline
\bottomrule
\end{longtable}

以下参考都是 2016 年文章.

\begin{enumerate}
\def\labelenumi{\arabic{enumi}.}
\tightlist
\item
  \href{https://www.gitbook.com/book/xiaomaer/compare-weex-to-react-native/details}{compare
  weex to react native}
\item
  \href{http://slides.com/ciyinhuang/weex\#/}{Weex 简介}
\item
  \href{http://zfx5130.me/blog/2016/09/15/Weex-\&-React-Native/}{Weex \&
  React Native}
\end{enumerate}

\subsection{评论摘抄}\label{ux8bc4ux8bbaux6458ux6284}

\begin{quote}
After a few days of experimentation, I realized Weex and its
documentation were not yet developed enough to for us to use to deliver
top-quality apps. This was my experience with Weex.
\href{https://www.bignerdranch.com/blog/is-vuejs-weex-a-suitable-alternative-to-react-native/}{Sam
Landfried, 2017.10.20, Is VueJS' Weex a Suitable Alternative to React
Native?}
\end{quote}

\section{Element}\label{element}

\url{https://github.com/ElemeFE/element}

\subsection{组件使用}\label{ux7ec4ux4ef6ux4f7fux7528}

\subsubsection{自定义表单校验}\label{ux81eaux5b9aux4e49ux8868ux5355ux6821ux9a8c}

\begin{lstlisting}
export default {
  data: function () {
    var checkVars = function (rule, value, callback) {
      if (!value) {
        callback(new Error('不能为空'));
      } else {
        callback();
      }
    };
    return {
      rules: {
        vars: [{
          required: true,
          trigger: 'change',
          validator: checkVars
        }]
      }
    }
  }
}
\end{lstlisting}

\subsection{兼容性}\label{ux517cux5bb9ux6027}

\subsubsection{IE 图标不显示}\label{ie-ux56feux6807ux4e0dux663eux793a}

可用文字替代伪元素中的内容.
