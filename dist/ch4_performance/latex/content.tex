\section{文件}\label{ux6587ux4ef6}

\subsection{图片}\label{ux56feux7247}

\subsubsection{图片格式的选择}\label{ux56feux7247ux683cux5f0fux7684ux9009ux62e9}

webp, gif, png, jpg, icon font

\subsubsection{icon font}\label{icon-font}

字体图片有两个优点: 矢量图放大后不失真; 起到图片精灵的作用,
减少图片请求次数.

图片转成字体文件, 作为矢量图, 常用于图标. 工具可用
\href{http://www.iconfont.cn/}{iconfont} 上传后生成 CSS 文件和字体.

\subsubsection{图片延迟加载/懒加载(lazy
load)}\label{ux56feux7247ux5ef6ux8fdfux52a0ux8f7dux61d2ux52a0ux8f7dlazy-load}

\paragraph{思路}\label{ux601dux8def}

延迟加载通常是将暂不需要的资源延后加载. 懒加载是延迟加载的一种,
即达到某个条件(或某个事件触发)时加载图片.

延迟加载可处理为, 当必要的资源加载完后再加载其余资源. 懒加载基本思路:

\begin{enumerate}
\def\labelenumi{\arabic{enumi}.}
\tightlist
\item
  暂存一张图片, 显示该默认图片.
\item
  显示图片的元素在可视区域时, 加载该图片.
\end{enumerate}

\paragraph{实例}\label{ux5b9eux4f8b}

具体到技术,
\href{https://h5.m.taobao.com/trip/home/index.html?_projVer=0.1.125}{飞猪H5}
的实现方法是:

\begin{lstlisting}[language=HTML]
<div class="base-bg base-bg-m regular-product__image___Bu73a" data-reactid=".0.$=1$trip_home_arbitrary_gate_product_0.0.$=1$regular_item_1.0.$=10">
    <div data-lazyloadid="lazyload_item_36" class="fade" style="opacity: 1;background-image: url(&quot;//gw.alicdn.com/tips/i3/638737216/TB2vvwZtVXXXXX0XXXXXXXXXXXX_!!638737216.jpg_400x400q75.jpg_.webp&quot;);"
        data-reactid=".0.$=1$trip_home_arbitrary_gate_product_0.0.$=1$regular_item_1.0.$=10.$=11" data-imageloaded="true"></div>
</div>
\end{lstlisting}

\begin{enumerate}
\def\labelenumi{\arabic{enumi}.}
\item
  父元素上设置默认的背景图片.

\begin{lstlisting}
.skin-yellow .base-bg {
  background: #f2f3f4 url(data:image/png;base64,iVBORw0KGgoAAAANSUhEUgAAALkAAABPCAMAAACAuJRqAAAAq1BMV…mgg7e+vIXHxHbzIMosU7LAtcvNOAUKpxf6kSUl8MPvAnj+AYRcPQeahlKYAAAAAElFTkSuQmCC) 50% no-repeat;
  background-size: auto .8rem;
}
\end{lstlisting}
\item
  子元素内联样式背景图片链接, 外链样式图片相关属性. 初始化时
  \lstinline!opacity: 0!, 并且不包含背景设置.

\begin{lstlisting}
opacity: 1;
background-image: url(//gw.alicdn.com/tips/i3/638737216/TB2vvwZtVXXXXX0XXXXXXXXXXXX_!!638737216.jpg_400x400q75.jpg_.webp);
\end{lstlisting}

\begin{lstlisting}
.base-bg>div {
  width: 100%;
  height: 100%;
  background-repeat: no-repeat;
  background-position: 50%;
  background-size: cover;
}
\end{lstlisting}
\item
  \textbf{满足条件时}, 设置子元素的背景图片(或者设置 img src 属性),
  然后标识已加载的标签一个属性(比如
  \lstinline!data-imageloaded="true"!), 如果是 img 标签, 加载后删除
  \lstinline!data-src!.
\end{enumerate}

\paragraph{关键点}\label{ux5173ux952eux70b9}

这里的\textbf{满足条件时}, 可用以下逻辑. 检查元素是否在可视区域,
可全局循环检查, 至于是否有性能问题, 待考察.

\begin{lstlisting}
loadIfVisible() // 如果在可视区域则加载
onScroll(loadIfVisible()); // 滚动事件触发时, 检查
\end{lstlisting}

判断元素是否在可视区域

\begin{lstlisting}
// 判断元素是否在可视区域
function isInView(obj) {
  var e = obj.getBoundingClientRect();
  return !(e.top > window.innerHeight || e.bottom < 0 || e.left > window.innerWidth || e.right < 0)
}
\end{lstlisting}

\paragraph{参考扩展}\label{ux53c2ux8003ux6269ux5c55}

\begin{enumerate}
\def\labelenumi{\arabic{enumi}.}
\tightlist
\item
  \href{https://stackoverflow.com/questions/123999/how-to-tell-if-a-dom-element-is-visible-in-the-current-viewport\#7557433}{stackoverflow,
  How to tell if a DOM element is visible in the current viewport?}
\item
  \href{https://developer.mozilla.org/zh-CN/docs/Web/API/Element/getBoundingClientRect}{mozilla,
  Element.getBoundingClientRect()}
\end{enumerate}

\section{体验优化}\label{ux4f53ux9a8cux4f18ux5316}

对页面性能的优化算起来都是体验优化, 这里主要指具有进一步提升性质的.
比如, 骨架屏实际上也可以用转圈圈来替代, 但其使用感受更好.

\subsection{骨架屏/Skeleton
Screen}\label{ux9aa8ux67b6ux5c4fskeleton-screen}

骨架屏指的是数据呈现之前, 一般用浅色的色条勾勒渲染后的轮廓.
相对通常的空白区域或者加 loading, 体验会好一些. 其次还起到了占位的作用.

\textbf{文章}

\begin{enumerate}
\def\labelenumi{\arabic{enumi}.}
\tightlist
\item
  \href{http://www.bestvist.com/2018/01/19/skeleton-screen/}{Skeleton
  Screen -- 骨架屏}
\item
  \href{https://www.sitepoint.com/how-to-speed-up-your-ux-with-skeleton-screens/}{How
  to Speed Up Your UX with Skeleton Screens}
\item
  \href{https://css-tricks.com/building-skeleton-screens-css-custom-properties/}{Building
  Skeleton Screens with CSS Custom Properties}
\end{enumerate}

\textbf{实例}

Ant Design 的 loading card, \url{https://ant.design/components/card/}
